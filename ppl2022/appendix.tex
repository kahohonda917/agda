\section*{付録}

\subsection*{A.補題}
ここでは、本文の補題\ref{Reduction2},\ref{IdkContext},\ref{TCompatible}について証明する。
%% \begin{lemma}
%%   $(M_l, M_r) \in R_{\tau}、M_l \rightarrow_{\beta} M_l'、M_r \rightarrow_{\beta} M_r'$のとき$(M_l', M_r') \in R_{\tau}$が成り立つ。
%% \label{Reduction1}
%% \end{lemma}
%% 簡約をしても論理関係は変わらないという補題である。\\
%%  \lbrack 証明 \rbrack\\
%%  $T=b$のとき$M_l' =_{\beta} M_r'$を示す。仮定より、$M_l' = _{\beta} M_l$。$(M_l,M_r)\in R_b$\ 。仮定より、$M_r = _{\beta} M_r'$なので$M_l' =_{\beta} M_r'$\ 。よって$(M_l',M_r')\in R_b$\\
%%  \\
%%  $T=\TypeArrow{\tau_2}{\tau_1}{\mu_{\alpha}}{\alpha}{\mu_{\beta}}{\beta}$のとき\\
%%  適切な$s,s'$について$(s,s')\in R_{\tau_2}$であると仮定する。また、$(\Lam{v_0}{\Lam{t_0}{K[]}}, \Lam{v_0}{\Lam{t_0}{K[]}})\in  K^k_{\tau_1(\mu_{\alpha})\alpha}$、$(t, t')\in T_{\mu_{\beta}}$とする。\\
%%  示したいことは、\\
%%  $(M_l, M_r) \in R_{\TypeArrow{\tau_2}{\tau_1}{\mu_{\alpha}}{\alpha}{\mu_{\beta}}{\beta}}、M_l \rightarrow_{\beta} M_l'、M_r \rightarrow_{\beta} M_r'$のとき$(M_l', M_r') \in R_{\TypeArrow{\tau_2}{\tau_1}{\mu_{\alpha}}{\alpha}{\mu_{\beta}}{\beta}}$\\
%%  $(M_l, M_r) \in R_{\TypeArrow{\tau_2}{\tau_1}{\mu_{\alpha}}{\alpha}{\mu_{\beta}}{\beta}}$の定義から、\\
%%  $(M_l\ s \ \LamP{v_0}{\Lam{t_0}{K[\Lam{t_1}{t_1}]}}\ \Cons{k}{t}, M_r\ s' \ \LamP{v_0}{\Lam{t_0}{K[\Lam{t_1}{\Cons{k}{t_1}}]}}\ t')\in R_{\beta}$\\
%%  $(M_l\ s \ \LamP{t_0}{K[\Lam{t_1}{t_1}]}\ t, M_r\ s' \ \LamP{t_0}{K[\Lam{t_1}{t_1}]}\ t')\in R_{\beta}$\\
%%  ここで、\\
%%  $M_l\ s \ \LamP{v_0}{\Lam{t_0}{K[\Lam{t_1}{t_1}]}}\ \Cons{k}{t} \rightarrow^* M_l'\ s \ \LamP{v_0}{\Lam{t_0}{K[\Lam{t_1}{t_1}]}}\ \Cons{k}{t}$\\
%%  $M_r\ s' \ \LamP{v_0}{\Lam{t_0}{K[\Lam{t_1}{\Cons{k}{t_1}}]}}\ t' \rightarrow^* M_r'\ s' \ \LamP{v_0}{\Lam{t_0}{K[\Lam{t_1}{\Cons{k}{t_1}}]}}\ t'$\\
%%  また、\\
%%  $M_l\ s \ \LamP{t_0}{K[\Lam{t_1}{t_1}]}\ t \rightarrow^* M_l'\ s \ \LamP{t_0}{K[\Lam{t_1}{t_1}]}\ t$\\
%%  $M_r\ s' \ \LamP{t_0}{K[\Lam{t_1}{t_1}]}\ t' \rightarrow^* M_r\ s' \ \LamP{t_0}{K[\Lam{t_1}{t_1}]}\ t'$\\
%%  帰納法の仮定より、\\
%%  $(M_l'\ s \ \LamP{v_0}{\Lam{t_0}{K[\Lam{t_1}{t_1}]}}\ \Cons{k}{t}, M_r'\ s' \ \LamP{v_0}{\Lam{t_0}{K[\Lam{t_1}{\Cons{k}{t_1}}]}}\ t')\in R_{\beta}$\\
%%  $(M_l'\ s \ \LamP{t_0}{K[\Lam{t_1}{t_1}]}\ t, M_r'\ s' \ \LamP{t_0}{K[\Lam{t_1}{t_1}]}\ t')\in R_{\beta}$\\
%%  前提と$(M_l', M_r') \in R_{\TypeArrow{\tau_2}{\tau_1}{\mu_{\alpha}}{\alpha}{\mu_{\beta}}{\beta}}$の定義から、$(M_l', M_r') \in R_{\TypeArrow{\tau_2}{\tau_1}{\mu_{\alpha}}{\alpha}{\mu_{\beta}}{\beta}}$が言える。

\setcounter{definition}{8}
\begin{lemma}
  $(M_l', M_r') \in R_{\tau}、M_l \rightarrow_{\beta} M_l'、M_r \rightarrow_{\beta} M_r'$のとき$(M_l, M_r) \in R_{\tau}$が成り立つ。
\end{lemma}
簡約をしても論理関係は変わらないという補題である。\\
 \lbrack 証明 \rbrack\\
 $T=b$のとき$M_l =_{\beta} M_r$を示す。仮定より、$M_l = _{\beta} M_l'$\ $(M_l',M_r')\in R_b$\ 仮定より、$M_r' = _{\beta} M_r$なので$M_l =_{\beta} M_r$\ 。よって$(M_l,M_r)\in R_b$\\
 \\
 $T=\TypeArrow{\tau_2}{\tau_1}{\mu_{\alpha}}{\alpha}{\mu_{\beta}}{\beta}$のとき\\
 適切な$s,s'$について$(s,s')\in R_{\tau_2}$であると仮定する。また、$(\Lam{v_0}{\Lam{t_0}{K[]}}, \Lam{v_0}{\Lam{t_0}{K[]}})\in  K^k_{\tau_1(\mu_{\alpha})\alpha}$、$(t, t')\in T_{\mu_{\beta}}$とする。\\
 示したいことは、\\
 $(M_l', M_r') \in R_{\TypeArrow{\tau_2}{\tau_1}{\mu_{\alpha}}{\alpha}{\mu_{\beta}}{\beta}}、M_l \rightarrow_{\beta} M_l'、M_r \rightarrow_{\beta} M_r'$のとき$(M_l, M_r) \in R_{\TypeArrow{\tau_2}{\tau_1}{\mu_{\alpha}}{\alpha}{\mu_{\beta}}{\beta}}$\\
 $(M_l', M_r') \in R_{\TypeArrow{\tau_2}{\tau_1}{\mu_{\alpha}}{\alpha}{\mu_{\beta}}{\beta}}$の定義から、\\
 $(M_l'\ s \ \LamP{v_0}{\Lam{t_0}{K[\Lam{t_1}{t_1}]}}\ \ConsP{k}{t}, M_r'\ s' \ \LamP{v_0}{\Lam{t_0}{K[\Lam{t_1}{\Cons{k}{t_1}}]}}\ t')\in R_{\beta}$\\
 $(M_l'\ s \ \LamP{t_0}{K[\Lam{t_1}{t_1}]}\ t, M_r'\ s' \ \LamP{t_0}{K[\Lam{t_1}{t_1}]}\ t')\in R_{\beta}$\\
 ここで、\\
 $M_l\ s \ \LamP{v_0}{\Lam{t_0}{K[\Lam{t_1}{t_1}]}}\ \ConsP{k}{t} \rightarrow^* M_l'\ s \ \LamP{v_0}{\Lam{t_0}{K[\Lam{t_1}{t_1}]}}\ \Cons{k}{t}$\\
 $M_r\ s' \ \LamP{v_0}{\Lam{t_0}{K[\Lam{t_1}{\Cons{k}{t_1}}]}}\ t' \rightarrow^* M_r'\ s' \ \LamP{v_0}{\Lam{t_0}{K[\Lam{t_1}{\Cons{k}{t_1}}]}}\ t'$\\
 また、\\
 $M_l\ s \ \LamP{t_0}{K[\Lam{t_1}{t_1}]}\ t \rightarrow^* M_l'\ s \ \LamP{t_0}{K[\Lam{t_1}{t_1}]}\ t$\\
 $M_r\ s' \ \LamP{t_0}{K[\Lam{t_1}{t_1}]}\ t' \rightarrow^* M_r\ s' \ \LamP{t_0}{K[\Lam{t_1}{t_1}]}\ t'$\\
 帰納法の仮定より、\\
 $(M_l\ s \ \LamP{v_0}{\Lam{t_0}{K[\Lam{t_1}{t_1}]}}\ \ConsP{k}{t}, M_r\ s' \ \LamP{v_0}{\Lam{t_0}{K[\Lam{t_1}{\Cons{k}{t_1}}]}}\ t')\in R_{\beta}$\\
 $(M_l\ s \ \LamP{t_0}{K[\Lam{t_1}{t_1}]}\ t, M_r\ s' \ \LamP{t_0}{K[\Lam{t_1}{t_1}]}\ t')\in R_{\beta}$\\
 前提と$(M_l, M_r) \in R_{\TypeArrow{\tau_2}{\tau_1}{\mu_{\alpha}}{\alpha}{\mu_{\beta}}{\beta}}$の定義から、$(M_l, M_r) \in R_{\TypeArrow{\tau_2}{\tau_1}{\mu_{\alpha}}{\alpha}{\mu_{\beta}}{\beta}}$が言える。
\\
\begin{lemma}
  $(k,k)\in T_{\mu_k}$の時、コンテキスト$\Lam{v_0}{\Lam{t_0}{K[f]}}\ =\ \Lam{v_0}{\Lam{t_0}{\Idk\ v_0\ (f\ t_0)}}$は$(\Lam{v_0}{\Lam{t_0}{K[]}}, \Lam{v_0}{\Lam{t_0}{K[]}})\in  K^k_{\beta(\mu_{id})\beta'}$を満たす。
\label{IdkContext}
\end{lemma}
この補題はcontrolとpromptのケースで使用する。
\\
\lbrack 証明 \rbrack\\
任意の$(V,V')\in R_{\beta}$と$(t_1,t_1')\in T_{\mu_{id}}$について\\
(A)\ $(\LamP{v_0}{\Lam{t_0}{K[\Lam{t_1}{t_1}]}}\ V\ (\Cons{k}{t_1}),
\qquad \LamP{v_0}{\Lam{t_0}{K[\Lam{t_1}{\Cons{k}{t_1}}]}}\ V\ t_1')\in R_{\beta'}$と\\
(B)\ $(\LamP{v_0}{\Lam{t_0}{K[\Lam{t_1}{t_1}]}}\ V\ t_1,
\qquad \LamP{v_0}{\Lam{t_0}{K[\Lam{t_1}{t_1}]}}\ V\ t_1')\in R_{\beta'}$を示せば良い。\\
\\
(A)コンテキストを代入して以下のようにする。\\
$(\LamP{v_0}{\Lam{t_0}{\Idk\ v_0\ t_0}}\ V\ (\Cons{k}{t_1}),
\LamP{v_0}{\Lam{t_0}{\Idk\ v_0\ (\Cons{k'}{t_0})}}\ V\ t_1')\in R_{\beta'}$\\
さらに補題\ref{Reduction2}より、簡約して$(\Idk\ V\ (\Cons{k}{t_1}),\Idk\ V'\ (\Cons{k'}{t_1'}))\in R_{\beta'}$を示す。\\
\\
(1)$t_1=\Bullet$の時\\
$(k\ V\ \Idt,k' V'\ \Idt)\in R_{\beta'}$を示す。\\
ここで$\mu_{id}=\Bullet$なので\textsf{id-cont-type}より$\beta=\beta'$となる。\\
$\mu_k = \Trail{\beta}{\beta'}{\mu_{id}}$となっているので\\
$(k,k')\in k_{\beta(\mu_{id})\beta'}$より$(V,V')\in R_{\beta}$と$(t_1,t_1')\in T_{\mu_{id}}$について\\
$(k\ V\ \Idt,k' V'\ \Idt)\in R_{\beta'}$となる。\\
\\
(2)$t_1 \neq \Bullet$の時\\
$(k\ V\ t,k' V'\ t')\in R_{\beta'}$を示す。\\
$\mu_k = \Trail{\beta}{\beta'}{\mu_{id}}$となっているので\\
$(k,k')\in k_{\beta(\mu_{id})\beta'}$より$(V,V')\in R_{\beta}$と$(t_1,t_1')\in T_{\mu_{id}}$について\\
$(k\ V\ t_1,k' V'\ t_1')\in R_{\beta'}$となり示せた。\\
\\
(B)\ コンテキストを代入して以下のようにする。\\
$(\LamP{v_0}{\Lam{t_0}{\Idk\ v_0\ t_0}}\ V\ t_1,
\LamP{v_0}{\Lam{t_0}{\Idk\ v_0\ t_0}}\ V\ t_1')\in R_{\beta'}$\\
さらに補題\ref{Reduction2}より、簡約して$(\Idk\ V\ t_1,\Idk\ V'\ t_1')\in R_{\beta'}$を示す。\\
\\
(1)$t_1=\Bullet$の時\\
$(V,V')\in R_{\beta'}$を示す。\\
ここで$\mu_{id}=\Bullet$なので\textsf{id-cont-type}より$\beta=\beta'$となるため示せた。\\
\\
(2)$t_1 \neq \Bullet$の時\\
$(t_1\ V\ \Idt,t_1' V'\ \Idt)\in R_{\beta'}$を示す。\\
\textsf{id-cont-type}より$\mu_{id}=\Trail{\beta}{\beta'}{\Bullet}$\\
よって$(V,V')\in R_{\beta}$と$(\Idt,\Idt)\in T_{\Bullet}$について$(t_1\ V\ \Idt,t_1' V'\ \Idt)\in R_{\beta'}$となる。

\begin{lemma}
  $\Compatible{\Trail{\tau_1}{\tau_1'}{\mu_1}}{\mu_2}{\mu_3}$かつ$(t_1, t_1')\in T_{\tau_1(\mu_1)\tau_1'}、(t_2, t_2')\in T_{\mu_2}、$のとき\\
  $(\Cons{t_1}{t_2}, \Cons{t_1'}{t_2'})\in T_{\mu_3}$
\end{lemma}
Tと\textsf{compatible}の関係についての補題である。メイン定理の\textsf{control}のケースを証明する際に使う。
\lbrack 証明 \rbrack\\
(1a) $\mu_2=\Bullet$のとき\\
$t_2=t_2'=\textsf{idt}$で $\Cons{t_1}{t_2}=t_1\ \Cons{t_1'}{t_2'}=t_1'$\\
\textsf{compatible}の定義から $\mu_3 = \Trail{\tau_1}{\tau_1'}{\mu_1}$\\
仮定より、$(t_1,t_1')\in T_{\mu_3}$が示せる。\\
\\
(1b) $\mu_2=\Trail{\tau_2}{\tau_2'}{\mu_2}$のとき\\
$\Compatible{\Trail{\tau_1}{\tau_1'}{\mu_1}}{\Trail{\tau_2}{\tau_2'}{\mu_2}}{\mu_3}$より$\mu_3 = \Trail{\tau_1}{\tau_1'}{\mu_3'}$かつ$\Compatible{\Trail{\tau_2}{\tau_2'}{\mu_2}}{\mu_3'}{\mu_1}$...(A)\\
仮定より、\\
$(t_1, t_1')\in T_{\tau_1(\mu_1)\tau_1'}...(1)、(t_2, t_2')\in T_{\tau_2(\mu_2)\tau_2'}$...(B)\\
示したいのは、$(\Cons{t_1}{t_2}, \Cons{t_1'}{t_2'})\in T_{\tau_1(\mu_3')\tau_1'}$\\これを\textsf{cons}の定義に従って展開すると次のようになる。\\
$(\Lam{v}{\Lam{t'}{t_1\,v\,(\Cons{t_2}{t'})}}, \Lam{v}{\Lam{t'}{t_1'\,v\,(\Cons{t_2'}{t'})}})\in T_{\tau_1(\mu_3')\tau_1'}$\\
Tの定義より、\\
$(V_1,V_1')\in R_{\tau_1}$...(2)と$(t_3,t_3')\in T_{\mu_3'}$...(C)を満たし、$(t_1\ V_1\ (\Cons{t_2}{t_3}),t_1'\ V_1'\ (\Cons{t_2'}{t_3'}))$となる。\\
(1),(2)より$(\Cons{t_2}{t_3},\Cons{t_2'}{t_3'})\in T_{\mu_1}$を示せば良い。\\
\\
次に(A),(B),(C)を利用して$(\Cons{t_2}{t_3},\Cons{t_2'}{t_3'})\in T_{\mu_1}$を示す。\\
(2a) $\mu_3'=\Bullet$のとき\\
$t_3=t_3'=\textsf{idt}$で $\Cons{t_2}{t_3}=t_2\ \Cons{t_2'}{t_3'}=t_2'$\\
\textsf{compatible}の定義から $\mu_1 = \Trail{\tau_2}{\tau_2'}{\mu_2}$\\
仮定より、$(t_2,t_2')\in T_{\mu_1}$が示せる。\\
\\
(2b) $\mu_3'=\Trail{\tau_3}{\tau_3'}{\mu_3'}$のとき\\
$\Compatible{\Trail{\tau_2}{\tau_2'}{\mu_2}}{\Trail{\tau_3}{\tau_3'}{\mu_3'}}{\mu_1}$より$\mu_1 = \Trail{\tau_2}{\tau_2'}{\mu_1'}$かつ$\Compatible{\Trail{\tau_3}{\tau_3'}{\mu_3'}}{\mu_1'}{\mu_2}$...(D)\\
仮定より、\\
$(t_2, t_2')\in T_{\tau_2(\mu_2)\tau_2'}...(3)、(t_3, t_3')\in T_{\tau_3(\mu_3)\tau_3'}$...(E)\\
示したいのは、$(\Cons{t_2}{t_3}, \Cons{t_2'}{t_3'})\in T_{\tau_2(\mu_1')\tau_2'}$\\これを\textsf{cons}の定義に従って展開すると次のようになる。\\
$(\Lam{v}{\Lam{t'}{t_2\,v\,(\Cons{t_3}{t'})}}, \Lam{v}{\Lam{t'}{t_2'\,v\,(\Cons{t_3'}{t'})}})\in T_{\tau_2(\mu_1')\tau_2'}$\\
Tの定義より、\\
$(V_2,V_2')\in R_{\tau_2}$...(4)と$(t_1,t_1')\in T_{\mu_1'}$...(F)を満たし、$(t_2\ V_2\ (\Cons{t_3}{t_1}),t_2'\ V_2'\ (\Cons{t_3'}{t_1'}))$となる。\\
(3),(4)より$(\Cons{t_3}{t_1},\Cons{t_3'}{t_1'})\in T_{\mu_2}$を示せば良い。\\
\\
次に(D),(E),(F)を利用して$(\Cons{t_3}{t_1},\Cons{t_3'}{t_1'})\in T_{\mu_2}$を示す。\\
(3a) $\mu_1'=\Bullet$のとき\\
$t_1=t_1'=\textsf{idt}$で $\Cons{t_3}{t_1}=t_3\ \Cons{t_3'}{t_1'}=t_3'$\\
\textsf{compatible}の定義から $\mu_2 = \Trail{\tau_3}{\tau_3'}{\mu_3}$\\
仮定より、$(t_3,t_3')\in T_{\mu_2}$が示せる。\\
\\
(3b) $\mu_1'=\Trail{\tau_1}{\tau_1'}{\mu_1'}$のとき\\
$\Compatible{\Trail{\tau_3}{\tau_3'}{\mu_3'}}{\Trail{\tau_1}{\tau_1'}{\mu_1'}}{\mu_2}$より$\mu_2 = \Trail{\tau_3}{\tau_3'}{\mu_2'}$かつ$\Compatible{\Trail{\tau_1}{\tau_1'}{\mu_1'}}{\mu_2'}{\mu_3'}$...(G)\\
仮定より、\\
$(t_3, t_3')\in T_{\tau_3(\mu_3')\tau_3'}...(5)、(t_1, t_1')\in T_{\tau_1(\mu_1')\tau_1'}$...(H)\\
示したいのは、$(\Cons{t_3}{t_1}, \Cons{t_3'}{t_1'})\in T_{\tau_3(\mu_2')\tau_3'}$\\これを\textsf{cons}の定義に従って展開すると次のようになる。\\
$(\Lam{v}{\Lam{t'}{t_3\,v\,(\Cons{t_1}{t'})}}, \Lam{v}{\Lam{t'}{t_3'\,v\,(\Cons{t_1'}{t'})}})\in T_{\tau_3(\mu_2')\tau_3'}$\\
Tの定義より、\\
$(V_3,V_3')\in R_{\tau_3}$...(6)と$(t_2,t_2')\in T_{\mu_2'}$...(I)を満たし、$(t_3\ V_3\ (\Cons{t_1}{t_2}),t_3'\ V_3'\ (\Cons{t_1'}{t_2'}))$となる。\\
(3),(4)より$(\Cons{t_1}{t_2},\Cons{t_1'}{t_2'})\in T_{\mu_3'}$を示せば良い。\\
(G),(H),(I)と帰納法の仮定より、$(\Cons{t_1}{t_2},\Cons{t_1'}{t_2'})\in T_{\mu_3'}$が示せる。
\\


\subsection*{B.証明}

\setcounter{definition}{7}
\begin{theorem}
  $x_i:\tau_i \vdash e:\tau \TrailsType{\alpha}{\beta} \TrailType{\beta}$であり、かつ各$v_i$が$\vdash v_i:\tau_i$かつ$(v_i,v_i') \in R_{\tau_i}$かつ任意の$\mu_k$型の$k$について$(k,k)\in T_{\mu_k}$が成り立ち、$(\Lam{v_0}{\Lam{t_0}{K[]}}, \Lam{v_0}{\Lam{t_0}{K[]}})\in  K^k_{\tau_1(\mu_{\alpha})\alpha}$を満たすような任意のコンテキスト$K$と、$(t, t') \in T_{\mu_{\beta}}$を満たす任意の$t, t'$とについて、\\
  (A)\ $(\rho \CPSTh{e} \LamP{v_0}{\Lam{t_0}{K[\Lam{t_1}{t_1}]}}\ \ConsP{k}{t}, \rho \CPSTh{e} \LamP{v_0}{\Lam{t_0}{K[\Lam{t_1}{\Cons{k}{t_1}}]}}\ t')\in R_{\beta}$\\
  (B)\ $(\rho \CPSTh{e} \LamP{v_0}{\Lam{t_0}{K[\Lam{t_1}{t_1}]}}\ t, \rho \CPSTh{e} \LamP{v_0}{\Lam{t_0}{K[\Lam{t_1}{t_1}]}}\ t')\in R_{\beta}$が成り立つ。\\
  ただし、(A)での$k$は$\Compatible{\mu_k}{\mu_{\alpha}}{\mu_{\alpha}}\ \Compatible{\mu_k}{\mu_{\beta}}{\mu_{\beta}}$を満たす。\\
\end{theorem}
\lbrack 証明 \rbrack\\\\
%%%%%%%%%%%%%%%%%%%%%%%%%%    VAR    %%%%%%%%%%%%%%%%%%%%%%%%%%    
\textbf{1.Varの場合}\\
型規則
\[
\begin{array}{c}
\infer[\TVar]
      {\JudgeTrail{\Gamma}{x}{\tau}{\TrailsType{\alpha}{\alpha}}
                                   {\TrailType{\alpha}}}
      {\Gamma(x)=\tau}
\end{array}
\]
\\
示すことは以下のようになる。\\
$x_i:\tau_i\ ,\ x:\tau\vdash e:\tau \TrailType{\alpha} \TrailType{\alpha}$であり、かつ各$v_i$が$\vdash v_i:\tau_i$かつ$(v_i,v_i') \in R_{\tau_i}\ \ (v,v')\in R_{\tau}$かつ$(\Lam{v_0}{\Lam{t_0}{K[]}}, \Lam{v_0}{\Lam{t_0}{K[]}})\in  K_{\tau_1(\mu_{\alpha})\alpha}$を満たすような任意のコンテキスト$K$と、$(t, t') \in T_{\mu_{\alpha}}$を満たす任意の$t, t'$について、\\
(A)\ $(\rho[x \rightarrow v] \CPSTh{x} \LamP{v_0}{\Lam{t_0}{K[\Lam{t_1}{t_1}]}}\ \ConsP{k}{t}, \rho[x \rightarrow v'] \CPSTh{x} \LamP{v_0}{\Lam{t_0}{K[\Lam{t_1}{\Cons{k'}{t_1}}]}}\ t')\in R_{\alpha}$\\
(ただし、$k$は任意の$\mu_k$型で、$\Compatible{\mu_k}{\mu_{\alpha}}{\mu_{\alpha}}$と$(k,k)\in T_{\mu_k}$を満たす)\\
  (B)\ $(\rho[x \rightarrow v] \CPSTh{x} \LamP{v_0}{\Lam{t_0}{K[\Lam{t_1}{t_1}]}}\ t, \rho[x \rightarrow v'] \CPSTh{x} \LamP{v_0}{\Lam{t_0}{K[\Lam{t_1}{t_1}]}}\ t')\in R_{\alpha}$が成り立つ。\\
\\
(A)CPS変換を展開して、\\
$(\LamP{v_0}{\Lam{t_0}{K[\Lam{t_1}{t_1}]}}\ v\ \ConsP{k}{t}, \LamP{v_0}{\Lam{t_0}{K[\Lam{t_1}{\Cons{k}{t_1}}]}}\ v'\ t')\in R_{\mu_{\alpha}}$を示す。\\
ここでコンテキスト$K$は$K_{\tau_1(\mu_{\alpha})\alpha}$を満たすので、$(V,V')\in R_{\tau}$を満たす$V,V'$と$(t_1, t_1')\in T_{\mu_{\alpha}}$を満たす$t_1, t_1'$について\\
$(\LamP{v_0}{\Lam{t_0}{K[\Lam{t_1}{t_1}]}}\ V\ \ConsP{k}{t_1}, \LamP{v_0}{\Lam{t_0}{K[\Lam{t_1}{\Cons{k}{t_1}}]}}\ V'\ t_1')\in R_{\mu_{\alpha}}$が言える。\\
前提より、$(v,v')\in R_{\tau}$ $(t, t') \in T_{\mu_{\alpha}}$なので示せた。\\
\\
(B)CPS変換を展開して、\\
$(\LamP{v_0}{\Lam{t_0}{K[\Lam{t_1}{t_1}]}}\ v\ t, \LamP{v_0}{\Lam{t_0}{K[\Lam{t_1}{t_1}]}}\ v'\ t')\in R_{\mu_{\alpha}}$を示す。\\
ここでコンテキスト$K$は$K_{\tau_1(\mu_{\alpha})\alpha}$を満たすので、$(V,V')\in R_{\tau}$を満たす$V,V'$と$(t_1, t_1')\in T_{\mu_{\alpha}}$を満たす$t_1, t_1'$について\\
$(\LamP{v_0}{\Lam{t_0}{K[\Lam{t_1}{t_1}]}}\ V\ t_1, \LamP{v_0}{\Lam{t_0}{K[\Lam{t_1}{t_1}]}}\ V'\ t_1')\in R_{\mu_{\alpha}}$が言える。\\
前提より、$(v,v')\in R_{\tau}$ $(t, t') \in T_{\mu_{\alpha}}$なので示せた。\\
\\
%%%%%%%%%%%%%%%%%%%%%%%%%%    FUN    %%%%%%%%%%%%%%%%%%%%%%%%%%
\textbf{2.Funの場合}\\
型規則
\[
\begin{array}{c}
\infer[\TFun]
      {\JudgeTrail{\Gamma}{\DLam{x}{e}}
                  {\ArrowTrailP{\tau_2}{\tau_1}
                               {\TrailsType{\alpha}{\beta}}
                               {\TrailType{\beta}}}
                  {\TrailType{\gamma}}
                  {\TrailType{\gamma}}}
      {\JudgeTrail{\Gamma,x:\tau_2}{e}{\tau_1}
                  {\TrailsType{\alpha}{\beta}}
                  {\TrailType{\beta}}}
\end{array}
\]
\\
示すことは以下のようになる。\\
$x_i:\tau_i \vdash \Lam{x}{e}:\ArrowTrailP{\tau_2}{\tau_1}{\TrailType{\alpha}}{\TrailType{\beta}}, \TrailType{\gamma}, \TrailType{\gamma}$であり、かつ各$v_i$が$\vdash v_i:\tau_i$かつ$(v_i,v_i') \in R_{\tau_i}$かつ$(\Lam{v_0}{\Lam{t_0}{K[]}}, \Lam{v_0}{\Lam{t_0}{K[]}})\in  K_{\ArrowTrailP{\tau_2}{\tau_1}{\TrailType{\alpha}}{\TrailType{\beta}} \TrailType{\gamma}}$を満たすような任意のコンテキスト$K$と、$(t_2, t_2') \in T_{\mu_{\gamma}}$を満たす任意の$t_2, t_2'$について、\\
(A)\ $(\rho \CPSTh{\Lam{x}{e}} \LamP{v_0}{\Lam{t_0}{K[\Lam{t_1}{t_1}]}}\ \ConsP{k}{t_2}, \rho \CPSTh{\Lam{x}{e}} \LamP{v_0}{\Lam{t_0}{K[\Lam{t_1}{\Cons{k}{t_1}}]}}\ t_2')\in R_{\gamma}$\\
(ただし、$k$は任意の$\mu_k$型で、$\Compatible{\mu_k}{\mu_{\gamma}}{\mu_{\gamma}}$と$(k,k)\in T_{\mu_k}$を満たす)\\
  (B)\ $(\rho \CPSTh{\Lam{x}{e}} \LamP{v_0}{\Lam{t_0}{K[\Lam{t_1}{t_1}]}}\ t_2, \rho \CPSTh{\Lam{x}{e}} \LamP{v_0}{\Lam{t_0}{K[\Lam{t_1}{t_1}]}}\ t_2')\in R_{\gamma}$が成り立つ。\\
\\
(A)CPS変換を展開して、\\
$(\LamP{v_0}{\Lam{t_0}{K[\Lam{t_1}{t_1}]}}
          \LamP{v}{\Lam{k'}{\Lam{t'}{\CPS{e}{\rho[\Change{x}{v}]}
           {\LamP{a}{\Lam{t''}
               {\App{\App{k'}{a}}{t''}}}}{t'}}}}\ \ConsP{k}{t_2},\\
  (\LamP{v_0}{\Lam{t_0}{K[\Lam{t_1}{\Cons{k}{t_1}}]}}
          \LamP{v}{\Lam{k'}{\Lam{t'}{\CPS{e}{\rho[\Change{x}{v}]}
           {\LamP{a}{\Lam{t''}
               {\App{\App{k'}{a}}{t''}}}}{t'}}}}\ t_2')\in R_{\gamma}$を示したい。\\
コンテキスト$K$は$K_{\ArrowTrailP{\tau_2}{\tau_1}{\TrailType{\alpha}}{\TrailType{\beta}} \TrailType{\gamma}}$を満たすので、\\
$(V,V')\in R_{\ArrowTrailP{\tau_2}{\tau_1}{\TrailType{\alpha}}{\TrailType{\beta}}}$を満たす$V,V'$ と $(t_2, t_2') \in T_{\mu_{\gamma}}$ を満たす $t_2, t_2'$ について\\
$(\LamP{v_0}{\Lam{t_0}{K[\Lam{t_1}{t_1}]}}\ V\ \ConsP{k}{t_2}, \LamP{v_0}{\Lam{t_0}{K[\Lam{t_1}{\Cons{k}{t_1}}]}}\ V'\ t_2')\in R_{\gamma}$となる。\\
つまり、\\
      $(\LamP{v}{\Lam{k'}{\Lam{t'}{\CPS{e}{\rho[\Change{x}{v}]}
           {\LamP{a}{\Lam{t''}
               {\App{\App{k'}{a}}{t''}}}}{t'}}}},\\
         \LamP{v}{\Lam{k'}{\Lam{t'}{\CPS{e}{\rho[\Change{x}{v}]}
           {\LamP{a}{\Lam{t''}
               {\App{\App{k'}{a}}{t''}}}}{t'}}}})\in R_{\ArrowTrailP{\tau_2}{\tau_1}{\TrailType{\alpha}}{\TrailType{\beta}}}$を示せば良い。\\
\\
$R_{\ArrowTrailP{\tau_2}{\tau_1}{\TrailType{\alpha}}{\TrailType{\beta}}}$の定義から、\\
任意の$(s,s')\in R_{\tau_2}$と$(\Lam{v_0}{\Lam{t_0}{K[]}}, \Lam{v_0}{\Lam{t_0}{K[]}})\in  K_{\tau_1(\mu_{\alpha})\alpha}$を満たすコンテキストと
$(t_3,t_3')\in T_{\mu_{\beta}}$を満たす$t_3, t_3'$と任意の$k_1$について次の2つを示せば良い。\\
(C)\ $(\CPS{e}{\rho[\Change{x}{s}]}
           {\LamP{a}{\Lam{t''}
               {\App{\App{\LamP{v_0}{\Lam{t_0}{K[\Lam{t_1}{t_1}]}}}{a}}{t''}}}}{\ConsP{k_1}{t_3}}),\\
  \qquad \CPS{e}{\rho[\Change{x}{s'}]}
           {\LamP{a}{\Lam{t''}
               {\App{\App{\LamP{v_0}{\Lam{t_0}{K[\Lam{t_1}{\Cons{k_1}{t_1}}]}}}{a}}{t''}}}}{t_3'}))\in R_{\beta}$\\
\\
(D)\ $(\CPS{e}{\rho[\Change{x}{s}]}
           {\LamP{a}{\Lam{t''}
               {\App{\App{\LamP{v_0}{\Lam{t_0}{K[\Lam{t_1}{t_1}]}}}{a}}{t''}}}}{t_3}),\\
  \qquad \CPS{e}{\rho[\Change{x}{s'}]}
           {\LamP{a}{\Lam{t''}
               {\App{\App{\LamP{v_0}{\Lam{t_0}{K[\Lam{t_1}{t_1}]}}}{a}}{t''}}}}{t_3'}))\in R_{\beta}$\\
\\
$e$に対しての帰納法の仮定より、\\
$x_i:\tau_i, x:\tau_2 \vdash e:\tau_1 \TrailType{\alpha} \TrailType{\beta}$であり、かつ各$v_i$が$\vdash v_i:\tau_i$かつ$(v_i,v_i') \in R_{\tau_i} (v, v')\in R_{\tau_2}$かつ$(\Lam{v_0}{\Lam{t_0}{K[]}}, \Lam{v_0}{\Lam{t_0}{K[]}})\in  K_{\tau_1(\mu_{\alpha})\alpha}$を満たすような任意のコンテキスト$K$と、$(t_3, t_3') \in T_{\mu_{\beta}}$を満たす任意の$t_3, t_3'$について、\\
(C)'\ $(\rho \CPSTh{e} \LamP{v_0}{\Lam{t_0}{K[\Lam{t_1}{t_1}]}}\ \ConsP{k}{t_3}, \rho \CPSTh{e} \LamP{v_0}{\Lam{t_0}{K[\Lam{t_1}{\Cons{k}{t_1}}]}}\ t_3')\in R_{\gamma}$\\
(ただし、$k$は任意の$\mu_k$型で、$\Compatible{\mu_k}{\mu_{\alpha}}{\mu_{\alpha}} \Compatible{\mu_k}{\mu_{\beta}}{\mu_{\beta}}$と$(k,k)\in T_{\mu_k}$を満たす)\\
(D)'\ $(\rho \CPSTh{e} \LamP{v_0}{\Lam{t_0}{K[\Lam{t_1}{t_1}]}}\ t_3, \rho \CPSTh{e} \LamP{v_0}{\Lam{t_0}{K[\Lam{t_1}{t_1}]}}\ t_3')\in R_{\gamma}$が成り立つ。\\
帰納法の仮定で(C),(D)を示すために、\\
$\Compatible{\mu_k}{\mu_{\alpha}}{\mu_{\alpha}} \Compatible{\mu_k}{\mu_{\beta}}{\mu_{\beta}}$を満たす事を調べる。\\
$\Compatible{\mu_k}{\mu_{\alpha}}{\mu_{\alpha}}$は前提より言える。\\
補題\ref{DiffCompatible}と$\Trails{\mu_{\alpha}}{\mu_{\beta}}$より、$\Compatible{\mu_{\beta}}{\mu_0}{\mu_{\alpha}}$で、補題\ref{ExtendCompatible'}より$\Compatible{\mu_k}{\mu_{\beta}}{\mu_{\beta}}$が言える。\\
次に、\\
$(\LamP{a}{\Lam{t''}
  {\App{\App{\LamP{v_0}{\Lam{t_0}{K[]}}}{a}}{t''}}},
  \LamP{a}{\Lam{t''}
    {\App{\App{\LamP{v_0}{\Lam{t_0}{K[]}}}{a}}{t''}}},)\in K_{\tau_1(\mu_{\alpha})\alpha}$を示す。\\
  \\
  $K$の定義より、\\
  $(V_1, V_1')\in R_{\tau_1}$を満たす$V_1,V_1'$と$(t,t')\in T_{\mu_{\alpha}}$を満たす$t,t'$と任意の$k$について\\
  (E)\ $(\LamP{a}{\Lam{t''}{\App{\App{\LamP{v_0}{\Lam{t_0}{K[\Lam{t_1}{t_1}]}}}{a}}{t''}}}\ V_1\ (\Cons{k}{t})),\\
  \qquad \LamP{a}{\Lam{t''}{\App{\App{\LamP{v_0}{\Lam{t_0}{K[\Lam{t_1}{\Cons{k}{t_1}}]}}}{a}}{t''}}}\ V_1\ t')\in R_{\alpha}$\\
  (F)\ $(\LamP{a}{\Lam{t''}{\App{\App{\LamP{v_0}{\Lam{t_0}{K[\Lam{t_1}{t_1}]}}}{a}}{t''}}}\ V_1\ t),\\
  \qquad \LamP{a}{\Lam{t''}{\App{\App{\LamP{v_0}{\Lam{t_0}{K[\Lam{t_1}{t_1}]}}}{a}}{t''}}}\ V_1\ t')\in R_{\alpha}$\\
  を示せば良い。\\
  \\
  補題\ref{Reduction2}より、簡約しても論理関係は変わらないので\\
  (E)\ $(\LamP{v_0}{\Lam{t_0}{K[\Lam{t_1}{t_1}]}}\ V_1\ (\Cons{k}{t}), \LamP{v_0}{\Lam{t_0}{K[\Lam{t_1}{\Cons{k}{t_1}}]}}\ V_1'\ t')$\\
  (F)\ $(\LamP{v_0}{\Lam{t_0}{K[\Lam{t_1}{t_1}]}}\ V_1\ t, \LamP{v_0}{\Lam{t_0}{K[\Lam{t_1}{t_1}]}}\ V_1'\ t')$\\
  を示せば良い。\\
  ここで、$(\Lam{v_0}{\Lam{t_0}{K[]}}, \Lam{v_0}{\Lam{t_0}{K[]}})\in  K_{\tau_1(\mu_{\alpha})\alpha}$を満たしており、\\
  $(V_1, V_1')\in R_{\tau_1} (t,t')\in T_{\mu_{\alpha}}$より(E),(F)が示せる。\\
  \\
  (B)も(A)と同様に示せる。
\\
\\  
%%%%%%%%%%%%%%%%%%%%%%%%%%    APP    %%%%%%%%%%%%%%%%%%%%%%%%%%
\textbf{3.Appの場合}\\
型規則  
\[
\begin{array}{c}
  \infer[\TApp]
      {\JudgeTrail{\Gamma}{\DApp{e_1}{e_2}}
                  {\tau_1}{\TrailsType{\alpha}{\delta}}{\TrailType{\delta}}}
      {\JudgeTrail{\Gamma}{e_1}
                  {\ArrowTrailP{\tau_2}{\tau_1}{\TrailsType{\alpha}{\beta}}
                                               {\TrailType{\beta}}}
                  {\TrailsType{\gamma}{\delta}}
                  {\TrailType{\delta}}
      &\JudgeTrail{\Gamma}{e_2}{\tau_2}{\TrailsType{\beta}{\gamma}}{\TrailType{\gamma}}}
\end{array}
\]
\\
示すことは以下のようになる。\\
$x_i:\tau_i\ \vdash e_1\ e_2:\tau_1 \TrailType{\alpha} \TrailType{\delta}$であり、かつ各$v_i$が$\vdash v_i:\tau_i$かつ$(v_i,v_i') \in R_{\tau_i}$かつ\\
$(\Lam{v_0}{\Lam{t_0}{K[]}}, \Lam{v_0}{\Lam{t_0}{K[]}})\in  K_{\tau_1(\mu_{\alpha})\alpha}$を満たすような任意のコンテキスト$K$と、$(t, t') \in T_{\mu_{\delta}}$を満たす任意の$t, t'$と任意の$k$について、\\
(A)\ $(\rho \CPSTh{e_1\ e_2} \LamP{v_0}{\Lam{t_0}{K[\Lam{t_1}{t_1}]}}\ (\Cons{k}{t}), \rho \CPSTh{e_1\ e_2} \LamP{v_0}{\Lam{t_0}{K[\Lam{t_1}{\Cons{k}{t_1}}]}}\ t')\in R_{\delta}$\\
(ただし、$k$は任意の$\mu_k$型で、$\Compatible{\mu_k}{\mu_{\alpha}}{\mu_{\alpha}} \Compatible{\mu_k}{\mu_{\delta}}{\mu_{\delta}}$と$(k,k)\in T_{\mu_k}$を満たす)\\
  (B)\ $(\rho \CPSTh{e_1\ e_2} \LamP{v_0}{\Lam{t_0}{K[\Lam{t_1}{t_1}]}}\ t, \rho \CPSTh{e_1\ e_2} \LamP{v_0}{\Lam{t_0}{K[\Lam{t_1}{t_1}]}}\ t')\in R_{\delta}$が成り立つ。\\
\\
まず、型規則から出てくる\textsf{compatible}を書き出しておく。\\
仮定より、$(1)\ \Compatible{\mu_k}{\mu_{\alpha}}{\mu_{\alpha}}\  (2)\ \Compatible{\mu_k}{\mu_{\delta}}{\mu_{\delta}}$\\
(2)と$\Trails{\mu_{\gamma}}{\mu_{\delta}}$と補題\ref{DiffCompatible}より、任意の$\mu_0$について$\Compatible{\mu_{\delta}}{\mu_0}{\mu_{\gamma}}$\\
これと(2)と補題\ref{ExtendCompatible'}より$(3)\ \Compatible{\mu_k}{\mu_{\gamma}}{\mu_{\gamma}}$\\
(3)と$\Trails{\mu_{\beta}}{\mu_{\gamma}}$と補題\ref{DiffCompatible}より、任意の$\mu_0$について$\Compatible{\mu_{\gamma}}{\mu_0}{\mu_{\beta}}$\\
これと(3)と補題\ref{ExtendCompatible'}より$(4)\ \Compatible{\mu_k}{\mu_{\beta}}{\mu_{\beta}}$\\
\\
それぞれCPS変換を展開して\\
  (A)\ $(\rho \CPSTh{e_1}
      \LamP{v_1}{\Lam{t_1}{\CPS{e_2}{\rho}
            {\LamP{v_2}{\Lam{t_2}
                {\AppP{\AppP{\AppP{v_1}{v_2}}
                    {\LamP{v_0}{\Lam{t_0}
                        {K[\Lam{t_1}{t_1}]}}}}{t_2}}}}{t_1}}}{(\Cons{k}{t})},\\
      \qquad \rho \CPSTh{e_1}
      \LamP{v_1}{\Lam{t_1}{\CPS{e_2}{\rho}
            {\LamP{v_2}{\Lam{t_2}
                {\AppP{\AppP{\AppP{v_1}{v_2}}
                    {\LamP{v_0}{\Lam{t_0}
                        {K[\Lam{t_1}{\Cons{k}{t_1}}]}}}}{t_2}}}}{t_1}}}{t'}\in R_{\delta}$\\
 \\
  (B)\ $(\rho \CPSTh{e_1}
      \LamP{v_1}{\Lam{t_1}{\CPS{e_2}{\rho}
            {\LamP{v_2}{\Lam{t_2}
                {\AppP{\AppP{\AppP{v_1}{v_2}}
                    {\LamP{v_0}{\Lam{t_0}
                        {K[\Lam{t_1}{t_1}]}}}}{t_2}}}}{t_1}}}{t},\\
      \qquad \rho \CPSTh{e_1}
      \LamP{v_1}{\Lam{t_1}{\CPS{e_2}{\rho}
            {\LamP{v_2}{\Lam{t_2}
                {\AppP{\AppP{\AppP{v_1}{v_2}}
                    {\LamP{v_0}{\Lam{t_0}
                        {K[\Lam{t_1}{t_1}]}}}}{t_2}}}}{t_1}}}{t'}\in R_{\delta}$\\
      \\
$e_1$についての帰納法の仮定より、\\
      $x_i:\tau_i\ \vdash e_1:\ArrowTrailP{\tau_2}{\tau_1}{\TrailType{\alpha}}{\TrailType{\beta}} \TrailType{\gamma} \TrailType{\delta}$であり
      $(\Lam{v_0}{\Lam{t_0}{K[]}}, \Lam{v_0}{\Lam{t_0}{K[]}})\in  K_{\ArrowTrailP{\tau_2}{\tau_1}{\TrailType{\alpha}}{\TrailType{\beta}} \TrailType{\gamma}}$を満たすような任意のコンテキスト$K$と、$(t, t') \in T_{\mu_{\delta}}$を満たす任意の$t, t'$について、\\
  (A)'\ $(\rho \CPSTh{e_1} \LamP{v_0}{\Lam{t_0}{K[\Lam{t_1}{t_1}]}}\ (\Cons{k}{t}), \rho \CPSTh{e_1} \LamP{v_0}{\Lam{t_0}{K[\Lam{t_1}{\Cons{k'}{t_1}}]}}\ t')\in R_{\delta}$\\
      (ただし、$k$は任意の$\mu_k$型で、$\Compatible{\mu_k}{\mu_{\alpha}}{\mu_{\alpha}} \Compatible{\mu_k}{\mu_{\beta}}{\mu_{\beta}}$と$(k,k)\in T_{\mu_k}$を満たす)\\
  (B)'\ $(\rho \CPSTh{e_1} \LamP{v_0}{\Lam{t_0}{K[\Lam{t_1}{t_1}]}}\ t, \rho \CPSTh{e_1} \LamP{v_0}{\Lam{t_0}{K[\Lam{t_1}{t_1}]}}\ t')\in R_{\delta}$が成り立つ。\\
      \\
ここで$k$の\textsf{compatible}関係については(1),(4)から言える。\\
そこで、\\
$(\LamP{v_1}{\Lam{t_1}{\CPS{e_2}{\rho}
            {\LamP{v_2}{\Lam{t_2}
                {\AppP{\AppP{\AppP{v_1}{v_2}}
                    {\LamP{v_0}{\Lam{t_0}
                        {K[]}}}}{t_2}}}}{t_1}}},\\
 \LamP{v_1}{\Lam{t_1}{\CPS{e_2}{\rho}
            {\LamP{v_2}{\Lam{t_2}
                {\AppP{\AppP{\AppP{v_1}{v_2}}
                    {\LamP{v_0}{\Lam{t_0}
                        {K[]}}}}{t_2}}}}{t_1}}})\in K_{\ArrowTrailP{\tau_2}{\tau_1}{\TrailType{\alpha}}{\TrailType{\beta}} \TrailType{\gamma}}$\\
 を示せば良い。\\
 \\
 $K_{\ArrowTrailP{\tau_2}{\tau_1}{\TrailType{\alpha}}{\TrailType{\beta}} \TrailType{\gamma}}$の定義から、\\
 $(V_1,V_1')\in R_{\ArrowTrailP{\tau_2}{\tau_1}{\TrailType{\alpha}}{\TrailType{\beta}}}$を満たす$V_1,V_1'$と$(t_3,t_3')\in T_{\mu_{\gamma}}$を満たす$t_3,t_3'$と任意の$k$について\\
 \\
 (C)\ $(\LamP{v_1}{\Lam{t_1}{\CPS{e_2}{\rho}
            {\LamP{v_2}{\Lam{t_2}
                {\AppP{\AppP{\AppP{v_1}{v_2}}
                    {\LamP{v_0}{\Lam{t_0}
                        {K[\Lam{t_1}{t_1}]}}}}{t_2}}}}{t_1}}}\ V_1\ (\Cons{k}{t_3}),\\
   \qquad    \LamP{v_1}{\Lam{t_1}{\CPS{e_2}{\rho}
            {\LamP{v_2}{\Lam{t_2}
                {\AppP{\AppP{\AppP{v_1}{v_2}}
                    {\LamP{v_0}{\Lam{t_0}
                        {K[\Lam{t_1}{\Cons{k}{t_1}}]}}}}{t_2}}}}{t_1}}}\ V_1'\ t_3')\in R_{\gamma}$\\
   \\
 (D)\ $(\LamP{v_1}{\Lam{t_1}{\CPS{e_2}{\rho}
            {\LamP{v_2}{\Lam{t_2}
                {\AppP{\AppP{\AppP{v_1}{v_2}}
                    {\LamP{v_0}{\Lam{t_0}
                        {K[\Lam{t_1}{t_1}]}}}}{t_2}}}}{t_1}}}\ V_1\ t_3,\\
   \qquad    \LamP{v_1}{\Lam{t_1}{\CPS{e_2}{\rho}
            {\LamP{v_2}{\Lam{t_2}
                {\AppP{\AppP{\AppP{v_1}{v_2}}
                    {\LamP{v_0}{\Lam{t_0}
                        {K[\Lam{t_1}{t_1}]}}}}{t_2}}}}{t_1}}}\ V_1'\ t_3')\in R_{\gamma}$\\
   \\
   また、補題\ref{Reduction2}より簡約して\\
 (C)\ $(\CPS{e_2}{\rho}
            {\LamP{v_2}{\Lam{t_2}
                {\AppP{\AppP{\AppP{V_1}{v_2}}
                    {\LamP{v_0}{\Lam{t_0}
                        {K[\Lam{t_1}{t_1}]}}}}{t_2}}}}{(\Cons{k}{t_3})},\\
   \qquad  \CPS{e_2}{\rho}
            {\LamP{v_2}{\Lam{t_2}
                {\AppP{\AppP{\AppP{V_1'}{v_2}}
                    {\LamP{v_0}{\Lam{t_0}
                        {K[\Lam{t_1}{\Cons{k}{t_1}}]}}}}{t_2}}}}{t_3'})\in R_{\gamma}$\\
            \\
 (D)\ $(\CPS{e_2}{\rho}
            {\LamP{v_2}{\Lam{t_2}
                {\AppP{\AppP{\AppP{V_1}{v_2}}
                    {\LamP{v_0}{\Lam{t_0}
                        {K[\Lam{t_1}{t_1}]}}}}{t_2}}}}{t_3},\\
   \qquad  \CPS{e_2}{\rho}
            {\LamP{v_2}{\Lam{t_2}
                {\AppP{\AppP{\AppP{V_1'}{v_2}}
                    {\LamP{v_0}{\Lam{t_0}
                        {K[\Lam{t_1}{t_1}]}}}}{t_2}}}}{t_3'})\in R_{\gamma}$\\
  を示せば良い。\\
  \\
  $e_2$についての帰納法の仮定より、\\
  $x_i:\tau_i\ \vdash e_2:\tau_2 \TrailsType{\beta}{\gamma} \TrailType{\gamma}$であり
  かつ$(\Lam{v_0}{\Lam{t_0}{K[]}}, \Lam{v_0}{\Lam{t_0}{K[]}})\in  K_{\tau_2(\mu_{\beta})\beta}$を満たすような任意のコンテキスト$K$と、$(t_3, t_3') \in T_{\mu_{\gamma}}$を満たす任意の$t, t'$について、\\
  (C)'\ $(\rho \CPSTh{e_2} \LamP{v_0}{\Lam{t_0}{K[\Lam{t_1}{t_1}]}}\ (\Cons{k_2}{t_3}), \rho \CPSTh{e_2} \LamP{v_0}{\Lam{t_0}{K[\Lam{t_1}{\Cons{k_2'}{t_1}}]}}\ t_3')\in R_{\gamma}$\\
  (ただし、$k$は任意の$\mu_k$型で、$\Compatible{\mu_k}{\mu_{\beta}}{\mu_{\beta}} \Compatible{\mu_k}{\mu_{\gamma}}{\mu_{\gamma}}$と$(k,k)\in T_{\mu_k}$を満たす)\\
  (D)'\ $(\rho \CPSTh{e_2} \LamP{v_0}{\Lam{t_0}{K[\Lam{t_1}{t_1}]}}\ t_3, \rho \CPSTh{e_2} \LamP{v_0}{\Lam{t_0}{K[\Lam{t_1}{t_1}]}}\ t_3')\in R_{\gamma}$が成り立つ。\\
  \\
  \\
  ここで$k$の\textsf{compatible}関係については(3),(4)から言える。\\
  帰納法の仮定を使って(C),(D)を導出するために以下を示す。\\
  $(\LamP{v_2}{\Lam{t_2}
                {\AppP{\AppP{\AppP{V_1}{v_2}}
                    {\LamP{v_0}{\Lam{t_0}
                        {K[]}}}}{t_2}}},\\
   \LamP{v_2}{\Lam{t_2}
                {\AppP{\AppP{\AppP{V_1}{v_2}}
                    {\LamP{v_0}{\Lam{t_0}
                        {K[]}}}}{t_2}}}\in K^{kk'}_{\tau_2(\mu_{\beta})\beta}$\\
   \\
   $K_{\tau_2(\mu_{\beta})\beta}$の定義より、以下を示す。\\
   $(V_2,V_2')\in R_{\tau_2}$を満たす任意の$V_2,V_2'$と $(t_4, t_4')\in T_{\mu_{\beta}}$を満たす任意の$t_4, t_4'$と任意の$k$について\\
   (E)\ $(\LamP{v_2}{\Lam{t_2}
                {\AppP{\AppP{\AppP{V_1}{v_2}}
                    {\LamP{v_0}{\Lam{t_0}
                        {K[\Lam{t_1}{t_1}]}}}}{t_2}}}\ V_2\ (\Cons{k}{t_4}),\\
   \qquad \LamP{v_2}{\Lam{t_2}
                {\AppP{\AppP{\AppP{V_1}{v_2}}
                    {\LamP{v_0}{\Lam{t_0}
                        {K[\Lam{t_1}{\Cons{k}{t_1}}]}}}}{t_2}}}\ V_2'\ t_4')$\\
   \\
   (F)\ $(\LamP{v_2}{\Lam{t_2}
                {\AppP{\AppP{\AppP{V_1}{v_2}}
                    {\LamP{v_0}{\Lam{t_0}
                        {K[\Lam{t_1}{t_1}]}}}}{t_2}}}\ V_2\ t_4,\\
   \qquad \LamP{v_2}{\Lam{t_2}
                {\AppP{\AppP{\AppP{V_1}{v_2}}
                    {\LamP{v_0}{\Lam{t_0}
                        {K[\Lam{t_1}{t_1}]}}}}{t_2}}}\ V_2'\ t_4')$\\
   \\
   補題\ref{Reduction2}より、簡約して\\
   (E)\ $(V_1\ V_2\ \LamP{v_0}{\Lam{t_0}{K[\Lam{t_1}{t_1}]}}\ (\Cons{k}{t_4}),\\
   \qquad V_1'\ V_2'\ \LamP{v_0}{\Lam{t_0}{K[\Lam{t_1}{\Cons{k'}{t_1}}]}}\ t_4')$\\
   \\
   (F)\ $(V_1\ V_2\ \LamP{v_0}{\Lam{t_0}{K[\Lam{t_1}{t_1}]}}\ t_4,\\
   \qquad V_1'\ V_2'\ \LamP{v_0}{\Lam{t_0}{K[\Lam{t_1}{t_1}]}}\ t_4')$\\
   を示せば良い。\\
   \\
   ここで、$V_1,V_1'$は$(V_1,V_1')\in R_{\ArrowTrailP{\tau_2}{\tau_1}{\TrailType{\alpha}}{\TrailType{\beta}}}$を満たしている。\\
   この定義より、$(V_2,V_2')\in R_{\tau_2}$と\\
   $(k,k')\in k_{\tau_1(\mu_{\alpha})\alpha}$かつ$(\Lam{v_0}{\Lam{t_0}{K[]}}, \Lam{v_0}{\Lam{t_0}{K[]}})\in  K_{\tau_1(\mu_{\alpha})\alpha}$を満たしているコンテキスト$K$と$(t_4, t_4')\in T_{\mu_{\beta}}$について、(E),(F)が導出できる。\\
\\
\\
%%%%%%%%%%%%%%%%%%%%%%%%%%    CONTROL    %%%%%%%%%%%%%%%%%%%%%%%%%%
\textbf{3.Controlの場合}\\
型規則
\[
\begin{array}{c}
  \infer[\TControl]
      {\JudgeTrail{\Gamma}{\DControl{k}{e}}{\tau}
                  {\TrailsType{\alpha}{\beta}}
                  {\TrailType{\beta}}}
      {\begin{array}{c}
       \IsIdTrail{\gamma}{\gamma'}{\MuId}\\
       \Compatible{\Trail{t_1}{t_2}{\mu_1}}{\mu_2}{\mu_0}\quad
       \Compatible{\mu_\beta}{\mu_0}{\mu_\alpha}\\
       \JudgeTrail{\Gamma,k:\ArrowTrail{\tau}{t_1}
                              {\TType{\mu_1}{t_2}}
                              {\TType{\mu_2}{\alpha}}}
                  {e}{\gamma}
                  {\TsType{\MuId}{\gamma'}{\Bullet}}
                  {\TType{\Bullet}{\beta}}
       \end{array}}
\end{array}
\]
\\
示すことは以下のようになる。\\
$x_i:\tau_i\ \vdash \Control{k}{e}:\tau \TrailType{\alpha} \TrailType{\beta}$であり、かつ各$v_i$が$\vdash v_i:\tau_i$かつ$(v_i,v_i') \in R_{\tau_i}$かつ\\
$(\Lam{v_0}{\Lam{t_0}{K[]}}, \Lam{v_0}{\Lam{t_0}{K[]}})\in  K_{\tau(\mu_{\alpha})\alpha}$を満たすような任意のコンテキスト$K$と、$(t_l, t_r) \in T_{\mu_{\beta}}$を満たす任意の$t, t'$と任意の$k$について、\\
(A)\ $(\rho \CPSTh{\Control{c}{e}} \LamP{v_0}{\Lam{t_0}{K[\Lam{t_1}{t_1}]}}\ (\Cons{k}{t_l}), \rho \CPSTh{\Control{c}{e}} \LamP{v_0}{\Lam{t_0}{K[\Lam{t_1}{\Cons{k}{t_1}}]}}\ t_r)\in R_{\beta}$\\
(ただし、$k$は任意の$\mu_k$型で、$\Compatible{\mu_k}{\mu_{\alpha}}{\mu_{\alpha}} \Compatible{\mu_k}{\mu_{\beta}}{\mu_{\beta}}$と$(k,k)\in T_{\mu_k}$を満たす)
(B)\ $(\rho \CPSTh{\Control{c}{e}}\ \LamP{v_0}{\Lam{t_0}{K[\Lam{t_1}{t_1}]}}\ t_l, \rho \CPSTh{\Control{c}{e}} \LamP{v_0}{\Lam{t_0}{K[\Lam{t_1}{t_1}]}}\ t_r)\in R_{\beta}$が成り立つ。\\
\\
(A)について、\\
CPS変換をすると、\\
$(\DLet{x'}
      {\Lam{v}{\Lam{k'}{\Lam{t'}
        {\DLet{t''}{\DAppend{\Cons{k}{t_l}}{\ConsP{k'}{t'}}}{\App{\App{\LamP{v_0}{\Lam{t_0}{K[\Lam{t_1}{t_1}]}}}{v}}{t''}}}}}}
      {\CPS{e}{\rho[\Change{c}{x'}]}{\Idk}{\Idt}},\\
  \DLet{x'}
      {\Lam{v}{\Lam{k'}{\Lam{t'}
        {\DLet{t''}{\DAppend{t_r}{\ConsP{k'}{t'}}}{\App{\App{\LamP{v_0}{\Lam{t_0}{K[\Lam{t_1}{\Cons{k}{t_1}}]}}}{v}}{t''}}}}}}
      {\CPS{e}{\rho[\Change{c}{x'}]}{\Idk}{\Idt}})\\
      \in R_{\beta}$となる。\\
\\
また、$\beta_{\Omega}$より$t''$を代入して\\
(A)\ $(\DLet{x'}
      {\Lam{v}{\Lam{k'}{\Lam{t'}
        {\App{\App{\LamP{v_0}{\Lam{t_0}{K[\Lam{t_1}{t_1}]}}}{v}}{\DAppend{\ConsP{k}{t_l}}{\ConsP{k'}{t'}}}}}}}
      {\CPS{e}{\rho[\Change{c}{x'}]}{\Idk}{\Idt}},\\
\qquad   \DLet{x'}
      {\Lam{v}{\Lam{k'}{\Lam{t'}
        {\App{\App{\LamP{v_0}{\Lam{t_0}{K[\Lam{t_1}{\Cons{k}{t_1}}]}}}{v}}{\DAppend{t_r}{\ConsP{k'}{t'}}}}}}}
      {\CPS{e}{\rho[\Change{c}{x'}]}{\Idk}{\Idt}})
      \in R_{\beta}$\\
\\
ここで、Agdaで示した\textsf{assoc}より$\DAppend{\ConsP{k}{t_l}}{\ConsP{k'}{t'}} =_{\beta} \Cons{k}{(\DAppend{t_l}{\ConsP{k'}{t'}}})$である。\\
$(\DLet{x'}
      {\Lam{v}{\Lam{k'}{\Lam{t'}
        {\App{\App{\LamP{v_0}{\Lam{t_0}{K[\Lam{t_1}{t_1}]}}}{v}}{\Cons{k}{(\DAppend{t_l}{\ConsP{k'}{t'}}})}}}}}
      {\CPS{e}{\rho[\Change{c}{x'}]}{\Idk}{\Idt}},\\
   \DLet{x'}
      {\Lam{v}{\Lam{k'}{\Lam{t'}
        {\App{\App{\LamP{v_0}{\Lam{t_0}{K[\Lam{t_1}{\Cons{k}{t_1}}]}}}{v}}{\DAppend{t_r}{\ConsP{k'}{t'}}}}}}}
      {\CPS{e}{\rho[\Change{c}{x'}]}{\Idk}{\Idt}})
      \in R_{\beta}$\\
      左右の$x'$の中身をそれぞれ$X,Y$とおくと、\\
      $(\CPS{e}{\rho[\Change{c}{X}]}{\Idk}{\Idt},\CPS{e}{\rho[\Change{c}{Y}]}{\Idk}{\Idt})\in R_{\beta}$を示せば良い。\\
\\
$e$に対しての帰納法の仮定より、\\
$x_i:\tau_i\ c:\tau \rightarrow t_1, (\mu_1) t_2 (\mu_2) \alpha \vdash e : \gamma (\mu_{id}) \gamma' (\Bullet) \beta$であり、かつ各$v_i$が$\vdash v_i:\tau_i$かつ$(v_i,v_i') \in R_{\tau_i}$で\\
$(\Lam{v_0}{\Lam{t_0}{K[]}}, \Lam{v_0}{\Lam{t_0}{K[]}})\in  K_{\gamma (\mu_{id}) \gamma'}$を満たすような任意のコンテキスト$K$と、$(t, t') \in T_{\Bullet}$を満たす任意の$t, t'$について、\\
$(\rho \CPSTh{e}\ \LamP{v_0}{\Lam{t_0}{K[\Lam{t_1}{t_1}]}}\ t, \rho \CPSTh{e} \LamP{v_0}{\Lam{t_0}{K[\Lam{t_1}{t_1}]}}\ t')\in R_{\beta}$が成り立つ。\\
\\
$\Lam{v_0}{\Lam{t_0}{K[f]}}\ =\ \Lam{v_0}{\Lam{t_0}{\Idk\ v_0\ (f\ t_0)}}$\\
上のコンテキストは$K_{\gamma (\mu_{id}) \gamma'}$を満たし(補題\ref{IdkContext})、$(t, t') \in T_{\Bullet}$より$t=t'=\Idt$なので
帰納法の仮定で(A)を導くには、$(X,Y)\in R_{\tau \rightarrow t_1, (\mu_1) t_2 (\mu_2) \alpha}$を示す。\\
$(\Lam{v}{\Lam{k'}{\Lam{t'}
    {\App{\App{\LamP{v_0}{\Lam{t_0}{K[\Lam{t_1}{t_1}]}}}{v}}{\Cons{k}{(\DAppend{t_l}{\ConsP{k'}{t'}}})}}}},\\
  \Lam{v}{\Lam{k'}{\Lam{t'}
      {\App{\App{\LamP{v_0}{\Lam{t_0}{K[\Lam{t_1}{\Cons{k}{t_1}}]}}}{v}}{\DAppend{t_r}{\ConsP{k'}{t'}}}}}})\in R_{\tau \rightarrow t_1, (\mu_1) t_2 (\mu_2) \alpha}$\\
\\
$R_{\tau \rightarrow t_1, (\mu_1) t_2 (\mu_2) \alpha}$の定義から\\
$(V,V')\in R_{\tau}$と$(\Lam{v_0}{\Lam{t_0}{K[]}}, \Lam{v_0}{\Lam{t_0}{K[]}})\in K_{t_1 (\mu_1) t_2}$を満たすコンテキスト$K_1$と$(t_{l2},t_{r2})\in T_{\mu_1}$と任意の$k$について\\
(C)\ $(X\ V\ \Lam{v_0}{\Lam{t_0}{K_1[\Lam{t_1}{t_1}]}}\ \ConsP{k}{t_{l2}},Y\ V'\ \Lam{v_0}{\Lam{t_0}{K_1[\Lam{t_1}{\Cons{k}{t_1}}]}}\ t_{r2})\in R_{\alpha}$\\
(D)\ $(X\ V\ \Lam{v_0}{\Lam{t_0}{K_1[\Lam{t_1}{t_1}]}}\ t_{l2},Y\ V'\ \Lam{v_0}{\Lam{t_0}{K_1[\Lam{t_1}{t_1}]}}\ t_{r2})\in R_{\alpha}$を示せば良い。\\
\\
(C)について\\
X,Yを代入して補題\ref{Reduction2}より簡約すると\\
$(\LamP{v_0}{\Lam{t_0}{K[\Lam{t_1}{t_1}]}}\ V\ \Cons{k}{(\DAppend{t_l}{\ConsP{\Lam{v_0}{\Lam{t_0}{K_1[\Lam{t_1}{t_1}]}}}{\ConsP{k}{t_{l2}}}}}),\\
\LamP{v_0}{\Lam{t_0}{K[\Lam{t_1}{\Cons{k'}{t_1}}]}}\ V'\ \DAppend{t_r}{\ConsP{\Lam{v_0}{\Lam{t_0}{K_1[\Lam{t_1}{\Cons{k'}{t_1}}]}}}{t_{r2}}})\in R_{\alpha}$を示す。\\
\\
$(\LamP{v_0}{\Lam{t_0}{K[]}},\LamP{v_0}{\Lam{t_0}{K[]}})\in K_{\tau(\mu_{\alpha})\alpha}$を満たしているから\\
$(V,V')\in R_{\tau}$と$(t, t')\in T_{\mu_{\alpha}}$と任意の$k$について\\
$(\LamP{v_0}{\Lam{t_0}{K[\Lam{t_1}{t_1}]}}\ V\ \ConsP{k}{t},
\Lam{v_0}{\Lam{t_0}{K_1[\Lam{t_1}{\Cons{k}{t_1}}]}}\ V'\ t')\in R_{\alpha}$\\
\\
そこで、\\
(E)\ $(\DAppend{t_l}{\ConsP{\Lam{v_0}{\Lam{t_0}{K_1[\Lam{t_1}{t_1}]}}}{\ConsP{k}{t_{l2}}}},
\DAppend{t_r}{\ConsP{\Lam{v_0}{\Lam{t_0}{K_1[\Lam{t_1}{\Cons{k}{t_1}}]}}}{t_{r2}}})\in T_{\mu_{\alpha}}$を示す。\\
\\
ここで、$\Compatible{\Trail{t_1}{t_2}{\mu_1}}{\mu_2}{\mu_0}\quad \Compatible{\mu_\beta}{\mu_0}{\mu_\alpha}$である事を考慮して、\\
次のように$trail$を考える。\\
%%%%%%%%%%%%%%%%%%%%%%%%%%%%%%%%trail設定%%%%%%%%%%%%%%%%%%%%%%%%%%%%%%%%%%%%
$\mu_0 = \Trail{t_1}{t_2}{\mu_0}\quad \mu_2 = \Trail{\epsilon}{\epsilon'}{\mu'}$\\
$\mu_{\beta} = \Trail{\delta}{\delta'}{\mu}\quad \mu_{\alpha} = \Trail{\delta}{\delta'}{\mu_{\alpha}}$\\
%%%%%%%%%%%%%%%%%%%%%%%%%%%%%%%%trail設定%%%%%%%%%%%%%%%%%%%%%%%%%%%%%%%%%%%%
\\
(E)の$::$を展開する。\\
(E)\ $(\DAppend{t_l}{\Lam{v_1}{\Lam{t_1'}{\LamP{v_0}{\Lam{t_0}{K_1[\Lam{t_1}{t_1}]}}}\ v_1\ \Cons{\ConsP{k}{t_{l2}}}{t_1'}}},\\
\qquad \DAppend{t_r}{\Lam{v_1}{\Lam{t_1'}{\LamP{v_0}{\Lam{t_0}{K_1[\Lam{t_1}{\Cons{k}{t_1}}]}}}\ v_1\ \ConsP{t_{r2}}{t_1'}}})\in T_{\mu_{\alpha}}$\\
次に$@$を展開する。また、$\mu_{\alpha} = \Trail{\delta}{\delta'}{\mu_{\alpha}}$なので\\
(E)\ $(\Lam{v_2}{\Lam{t_2'}{t_l\ v_2\ \Cons{\LamP{v_1}{\Lam{t_1'}{\LamP{v_0}{\Lam{t_0}{K_1[\Lam{t_1}{t_1}]}}}\ v_1\ \Cons{\ConsP{k}{t_{l2}}}{t_1'}}}{t_2'}}},\\
\qquad \Lam{v_2}{\Lam{t_2'}{t_r\ v_2\ \Cons{\LamP{v_1}{\Lam{t_1'}{\LamP{v_0}{\Lam{t_0}{K_1[\Lam{t_1}{\Cons{k}{t_1}}]}}}\ v_1\ \ConsP{t_{r2}}{t_1'}}}{t_2'}}})\in T_{\delta(\mu_{\alpha})\delta'}$\\
\\
$T_{\delta(\mu_{\alpha})\delta'}$の定義から\\
$(V_{delta},V_{delta}')\in R_{\delta}$を満たす$V,V'$と$(t_3, t_4)\in T_{\mu_{\alpha}}$を満たす$t_3, t_4$について\\
$(t_l\ V_{delta}\ \Cons{\LamP{v_1}{\Lam{t_1'}{\LamP{v_0}{\Lam{t_0}{K_1[\Lam{t_1}{t_1}]}}}\ v_1\ \Cons{\ConsP{k}{t_{l2}}}{t_1'}}}{t_3},\\
t_r\ V_{delta}'\ \Cons{\LamP{v_1}{\Lam{t_1'}{\LamP{v_0}{\Lam{t_0}{K_1[\Lam{t_1}{\Cons{k}{t_1}}]}}}\ v_1\ \ConsP{t_{r2}}{t_1'}}}{t_4})
\in R_{\delta'}$を示せば良い。\\
\\
ここで、$(t_l, t_r)\in T_{\mu_{\beta}}$つまり$(t_l, t_r)\in T_{\delta{\mu}\delta'}$なので定義より、\\
$(V_{delta},V_{delta}')\in R_{\delta}$を満たす$V,V'$とと$(t_{\mu}, t_{\mu}')\in T_{\mu}$を満たす$t_{\mu}, t_{\mu'}$について\\
$(t_l\ V_{delta}\ t_{\mu},t_r\ V_{delta}'\ t_{\mu}')\in R_{\delta'}$となるから\\
\\
(F)\ $(\Cons{\LamP{v_1}{\Lam{t_1'}{\LamP{v_0}{\Lam{t_0}{K_1[\Lam{t_1}{t_1}]}}}\ v_1\ \Cons{\ConsP{k}{t_{l2}}}{t_1'}}}{t_3},\\
\qquad \Cons{\LamP{v_1}{\Lam{t_1'}{\LamP{v_0}{\Lam{t_0}{K_1[\Lam{t_1}{\Cons{k}{t_1}}]}}}\ v_1\ \ConsP{t_{r2}}{t_1'}}}{t_4})
\in T_{\mu}$を示せば良い。\\
\\
ここで、(F)の$::$を展開し\\
(F)\ $(\Lam{v_2}{\Lam{t_2'}{\LamP{v_1}{\Lam{t_1'}{\LamP{v_0}{\Lam{t_0}{K_1[\Lam{t_1}{t_1}]}}}\ v_1\ \Cons{\ConsP{k}{t_{l2}}}{t_1'}}}\ v_2\ \ConsP{t_3}{t_2'}},\\
\qquad \Lam{v_2}{\Lam{t_2'}{\LamP{v_1}{\Lam{t_1'}{\LamP{v_0}{\Lam{t_0}{K_1[\Lam{t_1}{\Cons{k}{t_1}}]}}}\ v_1\ \ConsP{t_{r2}}{t_1'}}}\ v_2\ \ConsP{t_4}{t_2'}})\in T_{\mu}$となる。\\
\\
$\mu$について、\\
\setcounter{figure}{9}
\begin{figure}[h]
\[
\begin{array}{lcl}
  \Compatible{\mu_\beta}{\mu_0}{\mu_\alpha} &=& \Compatible{\Trail{\delta}{\delta'}{\mu}}{\Trail{t_1}{\mu_0}{t_2}}{\Trail{\delta}{\mu_{\alpha}}{\delta'}}\\
  &=& \Compatible{\Trail{t_1}{\mu_0}{t_2}}{\mu_{\alpha}}{\mu}
\end{array}
\]
\caption{\textsf{comptible}展開1}
\end{figure}
より$\mu=\Trail{t_1}{t_2}{\mu'}$と置くことができるので(F)は次のようになる。\\
\\
(F)\ $(\Lam{v_2}{\Lam{t_2'}{\LamP{v_1}{\Lam{t_1'}{\LamP{v_0}{\Lam{t_0}{K_1[\Lam{t_1}{t_1}]}}}\ v_1\ \Cons{\ConsP{k}{t_{l2}}}{t_1'}}}\ v_2\ \ConsP{t_3}{t_2'}},\\
\qquad \Lam{v_2}{\Lam{t_2'}{\LamP{v_1}{\Lam{t_1'}{\LamP{v_0}{\Lam{t_0}{K_1[\Lam{t_1}{\Cons{k}{t_1}}]}}}\ v_1\ \ConsP{t_{r2}}{t_1'}}}\ v_2\ \ConsP{t_4}{t_2'}})\in T_{t_1(\mu')t_2}$を示す。\\
定義より、\\
$(V_1,V_1')\in R_{t_1}$と$(t_{\mu_2},t_{\mu_2}')\in T_{\mu'}$について\\
(G)\ $(\LamP{v_0}{\Lam{t_0}{K_1[\Lam{t_1}{t_1}]}}\ V_1\ \Cons{\ConsP{k}{t_{l2}}}{\ConsP{t_3}{t_{\mu_2}}},\\
\qquad \LamP{v_0}{\Lam{t_0}{K_1[\Lam{t_1}{\Cons{k}{t_1}}]}}\ V_1'\ \Cons{t_{r2}}{\ConsP{t_4}{t_{\mu_2}'}})\in R_{t_2}$を示せば良い。\\
\\
ここで、\textsf{cons-assoc}より、$\Cons{\ConsP{k}{t_{l2}}}{\ConsP{t_3}{t_{\mu_2}}} =_{\beta} \Cons{k}{\ConsP{t_{l2}}{\ConsP{t_3}{t_{\mu_2}}}}$なので\\
(G)\ $(\LamP{v_0}{\Lam{t_0}{K_1[\Lam{t_1}{t_1}]}}\ V_1\ \Cons{k}{\ConsP{t_{l2}}{\ConsP{t_3}{t_{\mu_2}}}},\\
\qquad \LamP{v_0}{\Lam{t_0}{K_1[\Lam{t_1}{\Cons{k}{t_1}}]}}\ V_1'\ \Cons{t_{r2}}{\ConsP{t_4}{t_{\mu_2}'}})\in R_{t_2}$を示せば良い。\\
\\
ここで、$K_1$が$(\Lam{v_0}{\Lam{t_0}{K[]}}, \Lam{v_0}{\Lam{t_0}{K[]}})\in K_{t_1 (\mu_1) t_2}$を満たすので、\\
$(V_1,V_1')\in R_{t_1}$と$(t,t')\in T_{\mu_1}$と任意の$k$について\\
$(\LamP{v_0}{\Lam{t_0}{K_1[\Lam{t_1}{t_1}]}}\ V_1\ \ConsP{k}{t}, \LamP{v_0}{\Lam{t_0}{K_1[\Lam{t_1}{\Cons{k}{t_1}}]}}\ V_1'\ t)
\in R_{\tau_2}$が言える。\\
\\
つまり、(H)\ $(\ConsP{t_{l2}}{\ConsP{t_3}{t_{\mu_2}}},\Cons{t_{r2}}{\ConsP{t_4}{t_{\mu_2}'}})\in T_{\mu_1}$を示せば良い。
\\
ここで、\\
\begin{figure}[h]
\[
\begin{array}{lcl}
  \Compatible{\Trail{t_1}{t_2}{\mu_1}}{\mu_2}{\mu_0} &=& \Compatible{\Trail{t_1}{t_2}{\mu_1}}{\mu_2}{\Trail{t_1}{t_2}{\mu_0}}\\
  &=& \Compatible{\mu_2}{\mu_0}{\mu_1}
\end{array}
\]
\caption{\textsf{comptible}展開2}
\end{figure}\\
\\
また、図\ref{Compatible1}の続きから\\
\begin{figure}[h]
\[
\begin{array}{lcl}
  \Compatible{\Trail{t_1}{\mu_0}{t_2}}{\mu_{\alpha}}{\mu} &=& \Compatible{\Trail{t_1}{\mu_0}{t_2}}{\mu_{\alpha}}{\Trail{t_1}{t_2}{\mu'}}\\
  &=& \Compatible{\mu_{\alpha}}{\mu'}{\mu_0}
\end{array}
\]
\caption{\textsf{comptible}展開3}
\end{figure}\\
\\
(H)を示す。\\
$(t_3,t_4)\in T_{\mu_{\alpha}}\ (t_{\mu_2},t_{\mu_2}')\in T_{\mu'}$で図\ref{Compatible3}より$\Compatible{\mu_{\alpha}}{\mu'}{\mu_0}$\\
補題\ref{TCompatible}より$(\Cons{t_3}{t_{\mu_2}},\Cons{t_4}{t_{\mu_2}'})\in T_{\mu_0}$\\
また、$(t_{l2},t_{r2})\in T_{\mu_2}$で図\ref{Compatible2}より$\Compatible{\mu_2}{\mu_0}{\mu_1}$なので\\
補題\ref{TCompatible}より$(\Cons{t_{l2}}{\ConsP{t_3}{t_{\mu_2}}},\Cons{t_{r2}}{\ConsP{t_4}{t_{\mu_2}'}})\in T_{\mu_1}$となる。\\
これで(H)が示せた。\\
(B),(D)についても同様に示す。\\
\\
\\
%%%%%%%%%%%%%%%%%%%%%%%%%%    PROMPT    %%%%%%%%%%%%%%%%%%%%%%%%%%   
\textbf{4.Promptの場合}\\
型規則
\[
\begin{array}{c}
  \infer[\TPrompt]
      {\JudgeTrail{\Gamma}{\DReset{e}}{\tau}{\TrailsType{\alpha}{\alpha}}
                                            {\TrailType{\alpha}}}
      {\IsIdTrail{\beta}{\beta'}{\MuId}
      &\JudgeTrail{\Gamma}{e}{\beta}
                  {\TsType{\MuId}{\beta'}{\Bullet}}
                  {\TType{\Bullet}{\tau}}}
\end{array}
\]
\\
示すことは以下のようになる。\\
$x_i:\tau_i\ \vdash \Prompt{e}:\tau \TrailType{\alpha} \TrailType{\alpha}$であり、かつ各$v_i$が$\vdash v_i:\tau_i$かつ$(v_i,v_i') \in R_{\tau_i}$かつ\\
$(\Lam{v_0}{\Lam{t_0}{K[]}}, \Lam{v_0}{\Lam{t_0}{K[]}})\in  K^{kk'}_{\tau(\mu_{\alpha})\alpha}$を満たすような任意のコンテキスト$K$と、$(t, t') \in T_{\mu_{\alpha}}$を満たす任意の$t, t'$について、\\
(A)\ $(\rho \CPSTh{\Prompt{e}} \LamP{v_0}{\Lam{t_0}{K[\Lam{t_1}{t_1}]}}\ (\Cons{k}{t}), \rho \CPSTh{\Prompt{e}} \LamP{v_0}{\Lam{t_0}{K[\Lam{t_1}{\Cons{k}{t_1}}]}}\ t')\in R_{\alpha}$\\
(ただし、$k$は任意の$\mu_k$型で、$\Compatible{\mu_k}{\mu_{\alpha}}{\mu_{\alpha}}$と$(k,k)\in T_{\mu_k}$を満たす)
(B)\ $(\rho \CPSTh{\Prompt{e}}\ \LamP{v_0}{\Lam{t_0}{K[\Lam{t_1}{t_1}]}}\ t, \rho \CPSTh{\Prompt{e}} \LamP{v_0}{\Lam{t_0}{K[\Lam{t_1}{t_1}]}}\ t')\in R_{\alpha}$が成り立つ。\\
\\
それぞれのCPS変換を展開して、\\
(A)\ $(\DLet{v}{\CPS{e}{\rho}{\Idk}{\Idt}}{\SApp{\SApp{\LamP{v_0}{\Lam{t_0}{K[\Lam{t_1}{t_1}]}}}{v}}{(\Cons{k}{t})}},\\
\qquad   \DLet{v}{\CPS{e}{\rho}{\Idk}{\Idt}}{\SApp{\SApp{\LamP{v_0}{\Lam{t_0}{K[\Lam{t_1}{\Cons{k}{t_1}}]}}}{v}}{t'}})\in R_{\alpha}$ \\
\\
(B)\ $(\DLet{v}{\CPS{e}{\rho}{\Idk}{\Idt}}{\SApp{\SApp{\LamP{v_0}{\Lam{t_0}{K[\Lam{t_1}{t_1}]}}}{v}}{t}},\\
\qquad  \DLet{v}{\CPS{e}{\rho}{\Idk}{\Idt}}{\SApp{\SApp{\LamP{v_0}{\Lam{t_0}{K[\Lam{t_1}{t_1}]}}}{v}}{t'}})\in R_{\alpha}$ \\
\\
$(\Lam{v_0}{\Lam{t_0}{K[]}}, \Lam{v_0}{\Lam{t_0}{K[]}})\in  K_{\tau(\mu_{\alpha})\alpha}$より、\\
(C)\ $(\CPS{e}{\rho}{\Idk}{\Idt},\CPS{e}{\rho}{\Idk}{\Idt})\in R_{\tau}$を示せば良い。\\
\\
$e$に対する帰納法の仮定より、\\
$(\Lam{v_0}{\Lam{t_0}{K[]}}, \Lam{v_0}{\Lam{t_0}{K[]}})\in  K_{\beta(\mu_{id})\beta'}$を満たすような任意のコンテキスト$K$と、$(t, t') \in T_{\Bullet}$を満たす任意の$t, t'$について、\\
$(\rho \CPSTh{e}\ \LamP{v_0}{\Lam{t_0}{K[\Lam{t_1}{t_1}]}}\ t, \rho \CPSTh{e} \LamP{v_0}{\Lam{t_0}{K[\Lam{t_1}{t_1}]}}\ t')\in R_{\tau}$が成り立つ。\\
ここで、(C)を示すために、\\
補題\ref{IdkContext}より以下のコンテキストが$(\Lam{v_0}{\Lam{t_0}{K[]}}, \Lam{v_0}{\Lam{t_0}{K[]}})\in  K^{kk'}_{\beta(\mu_{id})\beta'}$を満たす。\\
$\Lam{v_0}{\Lam{t_0}{K[f]}}\ =\ \Lam{v_0}{\Lam{t_0}{\Idk\ v_0\ (f\ t_0)}}$\\
\\
また$(t, t') \in T_{\Bullet}$より$t=t'=\Idt$となる。\\
$(\rho \CPSTh{e}\ \LamP{v_0}{\Lam{t_0}{\Idk\ v_0\ t_0}}\ \Idt, \rho \CPSTh{e} \LamP{v_0}{\Lam{t_0}{\Idk\ v_0\ t_0}}\ \Idt)\in R_{\tau}$より示せた。
