\section{まとめと今後の方向性}
本稿では、control/prompt入り$\lambda$計算の体系において、
CPS変換の正当性を型付きで証明した。
CPSインタプリタとCPS変換をAgdaで定式化し、正当性の証明については
ほとんどを Agda で定式化したが、論理関係を使った同値関係を示す部分は
手による証明を行った。

我々のグループでは、4種類の限定継続演算子
(shift/reset \cite{DF1990,DF1992}、
control/prompt \cite{POPL88},
shift0/ reset0 \cite{materzok-subtyping},
control0/prompt0 \cite{gunter-cupto})
に対する型付き CPS 変換の正当性の証明を行っている。
本研究はその一環で、control/prompt に対する証明は一段落したと
考えているが、
これまでの shift/reset に対する証明に比べ複雑になっており、
証明の整理が必要と思われる。
ここでの成果は、同じく trail を使用する control0/prompt0 を含む
体系でも役に立つと期待される。
また、algebraic effects and handlers \cite{bauer-tutorial}
についても、同様に trail を使った CPS インタプリタが
作られており、
その正当性の証明を定式化する際にも役に立つ可能性がある。
