\section{CPS変換の正当性}
CPS変換の正当性を、一部を除いてAgdaで実装した。この節ではCPS変換の正当性を証明する。そのために証明に必要な補題から見ていく。
\subsection{補題}
\begin{lemma}[CPS変換と代入の可換性]\upshape
  任意の項$e_1$、値$v$、継続$k$、trail $t$について$\LamP{x}{e_1\,x}[v]\mapsto e_2$が成り立つとき、$\LamP{x}{\CPSTh{e_1\,x} \SAppS k \SAppS t}[\CPSTh{v}] \mapsto \CPSTh{e_2} \SAppS k \SAppS t$が成り立つ。
\label{eSubst}
\end{lemma}
これは$e_1$の中にある$x$を$v$に置き換えると$e_2$になる時、$e_1$のCPS変化後の項にある$x$を$v$のCPS変換後の値で置き換えると$e_2$のCPS変換結果になる、つまり代入とCPS変換の可換性を示している。代入規則によって場合分けを行い証明した。

\begin{lemma}[継続に関する代入演算]\upshape
  任意の項$e$、値$v$、trail $t$、schematicな継続$k$について\\
  $\LamP{x}{\CPSTh{e} \SAppS (k\, x) \SAppS t}[v] \mapsto \CPSTh{e} \SAppS (k\,v) \SAppS t$が成り立つ。
\label{kSubst}
\end{lemma}
これは$e$で場合分けをして帰納法の仮定を使う事で示せた。\\
継続$k$が schematic な場合(引数の構造を尊重する場合)、
$k$の中の$x$を$v$で置き換えられることを示している。
schematic \cite{DF1992}というのは、「引数の構造を尊重する」性質のことで、
本稿で扱う継続$k$は値とtrailの二つを受け取るため、
次の二式がこの性質を表す。\\
\[
\begin{array}{lcl}
  (k \SAppS x \SAppS t)[\Change{x}{v}] & = & k \SAppS v \SAppS t\\
  (k \SAppS v \SAppS x)[\Change{x}{t}] & = & k \SAppS v \SAppS t\\
\end{array}
\]

\begin{lemma}[継続の簡約に関する補題]\upshape
  任意の項$e_1$、値$v$、trail $t$、schematicな継続$k_1$について\\
  $(k_1 \SAppS \CPSTh{v} \SAppS t) \to (k_2 \SAppS \CPSTh{v} \SAppS t)$が成り立つとき、
  $\CPSTh{e}\SAppS k_1 \SAppS t_2 \to \CPSTh{e}\SAppS k_2 \SAppS t_2$が成り立つ。
\end{lemma}
継続の部分を同じ意味を持つものに置き換え証明を進めたい時に使う。$e$に関して場合分けを行い帰納法の仮定によって示せた。

\subsection{正当性の証明}
CPS変換前の項を簡約した後にCPS変換した項と、CPS変換後の項が$\beta$同値であること、
つまり以下の定理を示す。
% 定理4
\begin{theorem}[正当性の証明]\upshape
  任意の項$e$,$\, e'$について$e \to e'$が成り立つならば任意のtrail $t$とschematicな継続$k$に対して$\CPSTh{e}\SAppS k \SAppS t =_{\beta} \CPSTh{e'} \SAppS k \SAppS t$が成り立つ。
\label{TheoremCorrectEta}
\end{theorem}

ここで$=_{\beta}$は$\beta$同値である。
証明は、$e \to e'$の簡約基\textsf{RBeta}・\textsf{RFrame}・\textsf{RControl}・\textsf{RPrompt}で場合分けして実装した。\textsf{RControl}以外はCPS変換の定義と簡約規則を適用し、補題を使って変形することによって示すことができた。この部分はAgdaに載せることができた。\textsf{RControl}では、前述した補題の他にcontrolのための補題を立てる必要があった。また、trailに差分の情報を入れ、論理関係を使う必要もあった。これらについて次節以降で見ていく。


\subsection{controlのための補題}
定理4のcontrolのケースでは、任意のtrail $t$とschematicな継続$k$に対して\\
$\CPSTh{\Prompt{E_p[\Control{c}{e_1}]}}\SAppS k \SAppS t =_{\beta} \CPSTh{\Prompt{\App{\LamP{c}{e_1}}{\LamP{x}{E_p[x]}}}} \SAppS k \SAppS t$ が成り立つ事を示すということになる。そこでコンテキストで囲まれた左辺をCPS変換の定義によって展開できる形に変換するために次の補題を立てた。

\begin{lemma}[context-lemma]\upshape
  任意の項$e$、trail $t$、schematicな継続$k$について\\
  $\CPSTh{\Prompt{E_p[e]}}\,k\,t =_{\beta} \CPSTh{\App{\LamP{x}{E_p[x]}}{e}}\,k\,t$
\end{lemma}
context-lemmaは、評価文脈 $E_p$ に関する帰納法で証明することができる。
この補題を使って定理 \ref{TheoremCorrectEta} の control のケースを証明
するには、まず
$\CPSTh{\Prompt{E_p[\Control{c}{e_1}]}}\SAppS k \SAppS t =_{\beta} \CPSTh{\Prompt{\App{\LamP{c}{e_1}}{\LamP{x}{E_p[x]}}}} \SAppS k \SAppS t$
の両辺をCPS変換の定義によって展開していく。

\begin{figure}
\[
\begin{array}{cl}
 & \CPSTh{\Prompt{\App{\LamP{c}{e_1}}{\LamP{x}{E_p[x]}}}}\,k\,t\\
 = & let\ v = \CPSTh{\App{\LamP{c}{e_1}}{\LamP{x}{E_p[x]}}}(idk)(\bullet)\ in\ 
  k\SAppS v \SAppS t\\
  = & let\ v = \CPSTh{\LamP{c}{e_1}}\ \SLamP{v_1}{
    \SLam{t_1}{\CPSTh{\LamP{x}{E_p[x]}}\ \SLamP{v_2}{
        \SLam{t_2}{\DApp{\DApp{\DApp{v_1}{v_2}}
                    {\DLamP{v_2'}{\DLam{t''}
                        {\SApp{\SApp{idk}{v_2'}}{t''}}}}}{t_2}}}}t_1} \ \bullet\ \\
  & in\ k\SAppS v \SAppS t\\
  = & let\ v = \SLamP{v_1}{
    \SLam{t_1}{\CPSTh{\LamP{x}{E_p[x]}}\ \SLamP{v_2}{
        \SLam{t_2}{\DApp{\DApp{\DApp{v_1}{v_2}}
                    {\DLamP{v_2'}{\DLam{t''}
                        {\SApp{\SApp{idk}{v_2'}}{t''}}}}}{t_2}}}}t_1} \\
  & \SAppS
  \DLamP{ve}{\DLamP{k'e}{\DLamP{t'e}{\CPSTh{e}[\Change{c}{ve}]\
        \SLamP{a}{\SLamP{t''e}{k'e \DAppS a \DAppS t''e}}\ t'e}}} \SAppS \bullet\ in\
  k\SAppS v \SAppS t\\

  = & let\ v = \CPSTh{\LamP{x}{E_p[x]}}\ (\SLamS v_2.\ 
  \SLamS t_2.\ \\
   & \DApp{\DApp{\DApp{\DLamP{ve}{\DLamP{k'e}{\DLamP{t'e}{\CPSTh{e}[\Change{c}{ve}]\
        \SLamP{a}{\SLamP{t''e}{k'e \DAppS a \DAppS t''e}}\ t'e}}}}{v_2}}
                    {\DLamP{v_2'}{\DLam{t''}
                        {\SApp{\SApp{idk}{v_2'}}{t''}}}}}{t_2})\bullet\\
  & in\ k\SAppS v \SAppS t\\

  = & let\ v = (\SLamS v_2.\ 
  \SLamS t_2.\ 
  \DApp{\DApp{\DApp{\DLamP{ve}{\DLamP{k'e}{\DLamP{t'e}{\CPSTh{e}[\Change{c}{ve}]\
        \SLamP{a}{\SLamP{t''e}{k'e \DAppS a \DAppS t''e}}\ t'e}}}}{v_2}}
                    {\DLamP{v_2'}{\DLam{t''}
                        {\SApp{\SApp{idk}{v_2'}}{t''}}}}}{t_2})\\
  & \SAppS \DLamP{vE}{\DLam{k'E}{\DLam{t'E}{\CPSTh{E_p[x]}[\Change{x}{vE}]\
  \SLamP{aE}{\SLam{t''E}{k'E \DAppS aE \DAppS t''E}}\ t'E}}} \SAppS \bullet \ in\
  k\SAppS v \SAppS t\\

  = & let\ v =  \DLamP{ve}{\DLamP{k'e}{\DLamP{t'e}{\CPSTh{e}[\Change{c}{ve}]\
              \SLamP{a}{\SLamP{t''e}{k'e \DAppS a \DAppS t''e}}\ t'e}}}\DAppS\\
      & \DLamP{vE}{\DLam{k'E}{\DLam{t'E}{\CPSTh{E_p[x]}[\Change{x}{vE}]\
              \SLamP{aE}{\SLam{t''E}{k'E \DAppS aE \DAppS t''E}}\ t'E}}} \DAppS
                    \DLamP{v_2'}{\DLam{t''}
                      {\SApp{\SApp{idk}{v_2'}}{t''}}}\DAppS{\bullet}\\
                    & in\ k\SAppS v \SAppS t\\

  \to & let\ v = \CPSTh{e}[\Change{c}{\DLamP{vE}{\DLam{k'E}{\DLam{t'E}{\CPSTh{E_p[x]}[\Change{x}{vE}] \SLamP{aE}{\SLam{t''E}{k'E \DAppS aE \DAppS t''E}}\ t'E}}}}]\\
        & \SLamP{a}{\SLamP{t''e}{\DLamP{v_2'}{\DLam{t''}
              {\SApp{\SApp{idk}{v_2'}}{t''}}} \DAppS a \DAppS t''e}}\ \bullet
  \ in\
  k\SAppS v \SAppS t\\

  \to & let\ v = \CPSTh{e}[\Change{c}{\DLamP{vE}{\DLam{k'E}{\DLam{t'E}{\CPSTh{E_p[x]}[\Change{x}{vE}] \SLamP{aE}{\SLam{t''E}{k'E \DAppS aE \DAppS t''E}}\ t'E}}}}]\\
        & \SLamP{a}{\SLamP{t''e}{
              \SApp{\SApp{idk}{a}}{t''e}}}\ \bullet
  \ in\
  k\SAppS v \SAppS t\\

  \leftarrow & let\ v =\ let\ x'=
  \DLam{v'}{\DLam{k'}{\DLam{t'}{\CPSTh{E_p[x]}[\Change{x}{v'}]\
        \SLamP{a}{\SLam{t''}{k' \DAppS a \DAppS t''}}\ t'}}}\ \\
  & in\ \CPSTh{e}[\Change{c}{x'}]\
  idk\ \bullet \ in\ k\SAppS v \SAppS t\\
\end{array}
\]
\caption{簡約規則右辺の展開}
\label{ControlReRight}
\end{figure}


\begin{figure}
\[
\begin{array}{cl}
  & \CPSTh{\Prompt{E_p[\Control{c}{e_1}]}}\,k\,t\ \\
  = & let\ v=\CPSTh{E_p[\Control{c}{e_1}]}\ idk\ \bullet\ in\
  k\SAppS v \SAppS t\\

  \to & let\ v=\CPSTh{\App{\LamP{x}{E_p[x]}}{(\Control{c}{e_1})}}\ idk\ \bullet\ in\
  k\SAppS v \SAppS t\ (補題5より)\\

  = & let\ v=\CPSTh{\LamP{x}{E_p[x]}}\ (\SLamS v_1.\SLamS t_1.\CPSTh{\Control{c}{e_1}}\
  (\SLamS v_2.\SLamS t_2.\\ & \DAppP{\DAppP{\DAppP{v_1}{v_2}}
                    {\DLamP{v'}{\DLam{t''}
                        {\SApp{\SApp{idk}{v'}}{t''}}}}}{t_2})\ t_1\ ) \bullet\ in\ k\SAppS v \SAppS t\\

  = & let\ v=(\SLamS v_1.\SLamS t_1.\CPSTh{\Control{c}{e_1}}\
  (\SLamS v_2.\SLamS t_2.\DAppP{\DAppP{\DAppP{v_1}{v_2}}
                    {\DLamP{v'}{\DLam{t''}
                        {\SApp{\SApp{idk}{v'}}{t''}}}}}{t_2})\ t_1\ )\\
 & \SAppS \DLamP{v_3}{\DLam{k_2'}{\DLam{t_2'}{\CPSTh{E_p[x]}[\Change{x}{v_3}]\
        \SLamP{a}{\SLam{t_2''}{k_2' \DAppS a \DAppS t_2''}}\ t_2'}}} \SAppS \bullet \ in\
  k\SAppS v \SAppS t\\

  = & let\ v=(\SLamS v_1.\SLamS t_1.let\ x'= (\DLamS vF.\DLamS k'F.\DLam{t'F}{let\ t''F=\Append{t_1}{\Cons{k'F}{t'F}}}\\
  & in\ (\SLamS v_2.\SLamS t_2.\DAppP{\DAppP{\DAppP{v_1}{v_2}}
                    {\DLamP{v'}{\DLam{t''}
                        {\SApp{\SApp{idk}{v'}}{t''}}}}}{t_2})\SAppS vF \SAppS t''F)\ in\
            \CPSTh{e}[\Change{c}{x'}]\ idk\ \bullet)\\
  & \SAppS \DLamP{v_3}{\DLam{k_2'}{\DLam{t_2'}{\CPSTh{E_p[x]}[\Change{x}{v_3}]\
        \SLamP{a}{\SLam{t_2''}{k_2' \DAppS a \DAppS t_2''}}\ t_2'}}} \SAppS \bullet \ in\
            k\SAppS v \SAppS t\\
[3mm]
  & (中略)\\
[3mm]
  %% = & let\ v=let\ x'= (\DLamS vF.\DLamS k'F.\DLam{t'F}
  %%               {let\ t''F=\Append{\bullet}{\Cons{k'F}{t'F}}}\\
  %% & in\ (\SLamS v_2.\SLamS t_2.\DAppP{\DAppP{\DAppP
  %%     {\DLamP{v_3}{\DLam{k_2'}{\DLam{t_2'}{\CPSTh{E_p[x]}[\Change{x}{v_3}]\
  %%       \SLamP{a}{\SLam{t_2''}{k_2' \DAppS a \DAppS t_2''}}\ t_2'}}}}{v_2}}
  %%                   {\DLamP{v'}{\DLam{t''}
  %%                       {\SApp{\SApp{idk}{v'}}{t''}}}}}{t_2})\\
  %%               & \SAppS vF \SAppS t''F)\ in\ \CPSTh{e}[\Change{c}{x'}]\ idk\ \bullet \ in\
  %%               k\SAppS v \SAppS t\\

  %% = & let\ v=let\ x'= (\DLamS vF.\DLamS k'F.\DLam{t'F}
  %%               {let\ t''F=\Append{\bullet}{\Cons{k'F}{t'F}}}\\
  %% & in\ (\DAppP{\DAppP{\DAppP
  %%     {\DLamP{v_3}{\DLam{k_2'}{\DLam{t_2'}{\CPSTh{E_p[x]}[\Change{x}{v_3}]\
  %%       \SLamP{a}{\SLam{t_2''}{k_2' \DAppS a \DAppS t_2''}}\ t_2'}}}}{vF}}
  %%                   {\DLamP{v'}{\DLam{t''}
  %%                       {\SApp{\SApp{idk}{v'}}{t''}}}}}{t''F}))\ \\
  %%               & in\ \CPSTh{e}[\Change{c}{x'}]\ idk\ \bullet \ in\
  %%               k\SAppS v \SAppS t\\

 \to & let\ v=let\ x'= (\DLamS vF.\DLamS k'F.\DLam{t'F}
                {let\ t''F=\Append{\bullet}{\Cons{k'F}{t'F}}}\\
  & in\ (
     \CPSTh{E_p[x]}[\Change{x}{vF}]\
        \SLamP{a}{\SLam{t_2''}{\DLamP{v'}{\DLam{t''}
                        {\SApp{\SApp{idk}{v'}}{t''}}} \DAppS a \DAppS t_2''}}\ t''F
        )\ in\ \CPSTh{e}[\Change{c}{x'}]\ idk\ \bullet \ \\
        & in\
        k\SAppS v \SAppS t\\


  \to & let\ v=let\ x'= \DLamS v'.\DLamS k'.\DLamS t'.

     \CPSTh{E_p[x]}[\Change{x}{v'}]\
        idk\ \ConsP{k'}{t'}\\
                    & in\ \CPSTh{e}[\Change{c}{x'}]\ idk\ \bullet \ in\
                k\SAppS v \SAppS t\\

\end{array}
\]
\caption{簡約規則左辺の展開}
\label{ControlReLeft}
\end{figure}

図\ref{ControlReRight},\ref{ControlReLeft}に$\beta$同値を示したい両辺の展開を示した。それぞれの最終行に注目したい。両辺を見比べて形が異なっている部分を切り出すと次のようになる。\\
\[
\begin{array}{l}
  \DLam{v'}{\DLam{k'}{\DLam{t'}{\CPSTh{E_p[x]}[\Change{x}{v'}]\
        \SLamP{a}{\SLam{t''}{k' \DAppS a \DAppS t''}}\ t'}}}\\
  \DLamS v'.\,\DLamS k'.\,\DLamS t'.\,

     \CPSTh{E_p[x]}[\Change{x}{v'}]\
     \Idk\ \ConsP{k'}{t'}\\
     
\end{array}
\]
\\
ここで、上の式の継続部分について、$\SLam{a}{\SLam{t''}{k' \DAppS a \DAppS t''}}=\SLam{a}{\SLam{t''}{\Idk \DAppS a \DAppS \ConsP{k'}{t''}}}$と変換できる。なぜなら、$\SLam{a}{\SLam{t''}{\Idk \DAppS a \DAppS \ConsP{k'}{t''}}}$を$\Idk$の定義に従って展開すると
$\SLam{a}{\SLam{t''}{\ConsP{k'}{t''}} \DAppS a \DAppS \Idt}$であり、$t''=\Idt$の場合には$::$の定義より、$\SLam{a}{\SLam{t''}{k' \DAppS a \DAppS \Idt}}$で、$t''\neq \Idt$の場合には$\SLam{a}{\SLam{t''}{k' \DAppS a \DAppS t''}}$になるからである。

さらに$\Idk = \SLam{a}{\SLam{t''}{\Idk \SAppS a \SAppS t''}}$と表すことが出来ることから上の二式をかなり近い形に変換することができた。
\[
\begin{array}{l}
  \DLam{v'}{\DLam{k'}{\DLam{t'}{\CPSTh{E_p[x]}[\Change{x}{v'}]\
        \SLamP{a}{\SLam{t''}{\Idk \DAppS a \DAppS \ConsP{k'}{t''}}}\ t'}}}\\
  \DLamS v'.\,\DLamS k'.\,\DLamS t'.\,

     \CPSTh{E_p[x]}[\Change{x}{v'}]\
     \SLamP{a}{\SLam{t''}{\Idk \SAppS a \SAppS t''}}\ \ConsP{k'}{t'}\\
     
\end{array}
\]
そこで次の重要な補題を立てた。

\begin{lemma}[$k$の移動]\upshape
  任意の項$e$に対して以下が成り立つ。
\[
\begin{array}{cl}
& \DLam{v}{\DLam{k}{\DLam{t}{\CPSTh{e}[\Change{x}{v}] \SLamP{v_0}{\SLam{t_0}{\DApp{\Idk}{\DApp{v_0}{t_0}}}}\ \ConsP{k}{t}}}} \\
  =_{\beta} &
  \DLam{v}{\DLam{k}{\DLam{t}{\CPSTh{e}[\Change{x}{v}] \SLamP{v_0}{\SLam{t_0}{\SApp{\Idk}{\SApp{v_0}{\ConsP{k}{t_0}}}}}\ t}}}
\end{array}
\]
\label{KMove}
\end{lemma}
この補題では、trailに連なっている$k$の位置が継続に渡すtrailに移動しても$\beta$同値が成り立つ事を意味しており、Kameyama, Yonezawa \cite{KY2008}でも同様の補題を使っている。
これを補題として、定理4から呼び出すためにはtrailに関していくつかの補題が必要であった。左辺では$k$と$t$をtrailで繋げているのに対して、右辺では$k$と$t_0$を継続の中で繋げているため、この結合が型制約を満たしていることを保証する\textsf{compatible}関係が両辺で異なっている。補題\ref{KMove}を呼び出すためにはこの両方の\textsf{compatible}関係が必要となる。そこでtrailの型を変更して補題を立てた。次節で見ていく。

\subsection{trailの差分情報}
trailの型に差分情報を渡して次のようにAgdaで定義した。


\begin{verbatim}
data trails[_]_ (μα : trail) : trail → Set where
 [] : trails[ μα ] μα
 _::<_>_ : {τ1 τ2 : typ} {μ μβ μγ : trail} →
           (μk : trail) → (c : compatible μβ μk μγ) →
           (μs : trails[ μα ] μβ) →
           trails[ μα ] μγ
\end{verbatim}

trailsはtrailの差分情報を表している。$\mu_{\alpha}$をベースとして、空のtrailを受け取る場合の差分は$\mu_{\alpha}$と$\mu_{\alpha}$の差になる。$\mu_{\alpha}$に1つ以上のtrailを連ねてできた$\mu_s$に空ではない$\mu_k$を連ねる場合は、$\mu_{\beta}$と$\mu_k$を繋げたら型が$\mu_{\gamma}$になる事保証する\textsf{compatible}を同時に渡して$\mu_{\alpha}$から$\mu_{\gamma}$の差分を返している。
このtrailsを使い、持っているtrailsの型から\textsf{compatible}関係を導くことが出来る補題を立てた。これで補題6をメイン定理から呼び出す準備が整った。
\begin{lemma}[diff-compatible]\upshape
  $\mu_{\beta}$に一つ以上のtrailが連なって$\mu_{\alpha}$になったなら、
 あるtrail $\mu_0$が存在して
 $\Compatible{\mu_{\beta}}{\mu_0}{\mu_{\alpha}}$が成り立つ。
\end{lemma}


\subsection{論理関係の定義}
補題\ref{KMove}を示すにあたって、Appのケースでは2つの部分項の再帰だけではなく、関数部分の項が実行された結果の値$v$においても補題を満たさなければならないと分かった。
つまり、単純な項に関する帰納法では証明はできない。
そこで、これを示すために論理関係を定義した。
論理関係のメイン定理は、環境を明示的に表現する必要があり、環境部分を
メタレベル(Agda のレベル)で扱うPHOAS \cite{chlipala-phoas}
で実装するのは難しい。
そこで補題\ref{KMove}の証明に関しては手証明で進めることにする。
補題\ref{KMove}を証明できるように定義した論理関係を下に示す。

\[
\begin{array}{l}
  (M, M')\in R_d \ \Longleftrightarrow\ M =_{\beta}M'\\
  (M, M')\in R_{\tau_2 \rightarrow \tau_1 (\mu_{\alpha}) \alpha (\mu_{\beta})\beta}\\
  \begin{array}{cl}
  \Longleftrightarrow &
  \forall (V, V')\in R_{\tau_2}.\\
& \forall\  k\\
& \forall (\Lam{v_0}{\Lam{t_0}{K[]}}, \Lam{v_0}{\Lam{t_0}{K[]}})\in  K^k_{\tau_1(\mu_{\alpha})\alpha}.\\
& \forall (t, t')\in T_{\mu_{\beta}}.\\
& (M\ V \ \Lam{v_0}{\Lam{t_0}{K[\Lam{t_1}{t_1}]}}\ \ConsP{k}{t}, M'\ V' \ \Lam{v_0}{\Lam{t_0}{K[\Lam{t_1}{\Cons{k}{t_1}}]}}\ t')\in R_{\beta}\\
  & (M\ V \ \Lam{t_0}{K[\Lam{t_1}{t_1}]}\ t, M'\ V' \ \Lam{t_0}{K[\Lam{t_1}{t_1}]}\ t')\in R_{\beta}\\
  \end{array}
\end{array}
\]
コンテキストについて
\[
\begin{array}{l}
  (\Lam{v_0}{\Lam{t_0}{K[]}}, \Lam{v_0}{\Lam{t_0}{K[]}}) \in K^k_{\tau_1(\mu_{\alpha})\alpha}\\
  \begin{array}{cl}
  \Longleftrightarrow
& \forall (V, V')\in R_{\tau_1}.\\
& \forall (t, t')\in T_{\mu_{\alpha}}.\\
& (\Lam{v_0}{\Lam{t_0}{K[\Lam{t_1}{t_1}]}}\ V\ \ConsP{k}{t}, \Lam{v_0}{\Lam{t_0}{K[\Lam{t_1}{\Cons{k}{t_1}}]}}\ V'\ t')\in R_{\alpha}\\
& (\Lam{v_0}{\Lam{t_0}{K[\Lam{t_1}{t_1}]}}\ V\ t, \Lam{v_0}{K[\Lam{t_1}{t_1}]}\ V'\ t')\in R_{\alpha}
  \end{array}
\end{array}
\]
trailについて
\[
\begin{array}{lcl}
  (t, t')\in T_{bullet} &\Longleftrightarrow&  t=t'=idt\\
  (t, t')\in T_{\tau_1(\mu_{\alpha})\alpha} &\Longleftrightarrow&
  \forall (V, V')\in R_{\tau_1}.\\
  & & \forall (t_2, t_2')\in T_{\mu_{\alpha}}.\\
  & & (t\ V\ t_2, t'\ V'\ t_2')\in R_{\alpha}\\
\end{array}
\]

この論理関係は、ふたつの値が同じように振る舞うことを示すものである。
特に、関数型を持った場合、同じように振る舞う引数を受け取ったら、関数呼
び出しをした結果も同じように振る舞うことを示している。
これを使うと、関数呼び出しのケースで関数部分を実行した結果についても良
い性質を満たすことを言えるようになる。
$R$ 関係はふたつの結論がある。
最初のものが $k::$ を移動できることを示すものだが、この性質を証明する
際、control と prompt のケースで同一の継続、trail について同値関係が成
り立つことが必要になるため、ふたつ目の結論を用意した。
上手に定式化すると両者をひとつにまとめられると思われるが、それは今後の
課題である。

\subsection{手証明のメイン定理と補題}
ここで、環境が入った次の定理(論理関係のメイン定理)を提示する。
定理\ref{HandMain}を使う事で補題\ref{KMove}を系の形で示すことが出来る。
今後の節では定理\ref{HandMain}の証明、補題\ref{KMove}の証明の順に見ていく。
\begin{theorem}
  $x_i:\tau_i \vdash e:\tau \TrailsType{\alpha}{\beta} \TrailType{\beta}$であり、かつ各$v_i$が$\vdash v_i:\tau_i$かつ$(v_i,v_i') \in R_{\tau_i}$かつ任意の$\mu_k$型の$k$について$(k,k)\in T_{\mu_k}$が成り立ち、$(\Lam{v_0}{\Lam{t_0}{K[]}}, \Lam{v_0}{\Lam{t_0}{K[]}})\in  K^k_{\tau_1(\mu_{\alpha})\alpha}$を満たすような任意のコンテキスト$K$と、$(t, t') \in T_{\mu_{\beta}}$を満たす任意の$t, t'$とについて、\\
  (A)\ $(\rho \CPSTh{e} \LamP{v_0}{\Lam{t_0}{K[\Lam{t_1}{t_1}]}}\ \ConsP{k}{t}, \rho \CPSTh{e} \LamP{v_0}{\Lam{t_0}{K[\Lam{t_1}{\Cons{k}{t_1}}]}}\ t')\in R_{\beta}$\\
  (B)\ $(\rho \CPSTh{e} \LamP{v_0}{\Lam{t_0}{K[\Lam{t_1}{t_1}]}}\ t, \rho \CPSTh{e} \LamP{v_0}{\Lam{t_0}{K[\Lam{t_1}{t_1}]}}\ t')\in R_{\beta}$が成り立つ。\\
  ただし、(A)での$k$は$\Compatible{\mu_k}{\mu_{\alpha}}{\mu_{\alpha}}\ \Compatible{\mu_k}{\mu_{\beta}}{\mu_{\beta}}$を満たす。\\
\label{HandMain}
\end{theorem}
定理\ref{HandMain}を示すにあたって必要となった補題を下に挙げる。
\begin{lemma}[簡約と論理関係の保存]
  $(M_l', M_r') \in R_{\tau}、M_l \rightarrow_{\beta} M_l'、M_r \rightarrow_{\beta} M_r'$のとき$(M_l, M_r) \in R_{\tau}$が成り立つ。
\label{Reduction2}
\end{lemma}
\begin{lemma}
  $(k,k)\in T_{\mu_k}$の時、コンテキスト$\Lam{v_0}{\Lam{t_0}{K[f]}}=\Lam{v_0}{\Lam{t_0}{\Idk\ v_0\ (f\ t_0)}}$は\\
  $(\Lam{v_0}{\Lam{t_0}{K[]}}, \Lam{v_0}{\Lam{t_0}{K[]}})\in  K^k_{\beta(\mu_{id})\beta'}$を満たす。
\label{IdkContext}
\end{lemma}
これは該当のコンテキストが$K$の条件を満たす事を示しており、$\Idk$が登場するcontrolとpromptのケースで使う。

controlのケースを示すためにさらに次の二つの補題が必要となった。
\begin{lemma}\upshape
  $\Compatible{\Trail{\tau_1}{\tau_1'}{\mu_1}}{\mu_2}{\mu_3}$かつ$(t_1, t_1')\in T_{\tau_1(\mu_1)\tau_1'}、(t_2, t_2')\in T_{\mu_2}、$のとき\\
  $(\Cons{t_1}{t_2}, \Cons{t_1'}{t_2'})\in T_{\mu_3}$
\label{TCompatible}
\end{lemma}
補題\ref{TCompatible}は、論理関係の$T$の定義と持っている\textsf{compatible}関係から新たな$T$の関係を示すものである。
このように $::$ や $@$ を使って trail を結合するたびに、その
\textsf{compatible} 関係を示す必要がある。

\begin{lemma}\upshape
  任意のtrail\ $t_1,t_2,t_3$について以下が成り立つ。\\
  $\Cons{\ConsP{t_1}{t_2}}{t_3} =_{\beta} \Cons{t_1}{\ConsP{t_2}{t_3}}$
\label{ConsAssoc}
\end{lemma}
補題\ref{ConsAssoc}は\textsf{cons-assoc}としてAgdaに実装済みである。trailの形で場合分けを行い、帰納法の仮定を使う事で示せた。
補題\ref{ConsAssoc}以外については手で証明を行ったため付録に記載する。

\subsection{定理\ref{HandMain}の証明}
定理\ref{HandMain}は$e$についての帰納法で証明した。
証明の核心であるcontrolのケースについて下に示す。その他のケースについては付録に記載している。\\
\\
示すことは以下のようになる。\\
$x_i:\tau_i\ \vdash \Control{k}{e}:\tau \TrailType{\alpha} \TrailType{\beta}$であり、かつ各$v_i$が$\vdash v_i:\tau_i$かつ$(v_i,v_i') \in R_{\tau_i}$かつ\\
$(k,k')\in k_{\tau(\mu_{\alpha})\alpha}$かつ$(\Lam{v_0}{\Lam{t_0}{K[]}}, \Lam{v_0}{\Lam{t_0}{K[]}})\in  K^{kk'}_{\tau(\mu_{\alpha})\alpha}$を満たすような任意のコンテキスト$K$と、$(t_l, t_r) \in T_{\mu_{\beta}}$を満たす任意の$t, t'$と任意の$k$について、\\
  (A)\ $(\rho \CPSTh{\Control{c}{e}} \LamP{v_0}{\Lam{t_0}{K[\Lam{t_1}{t_1}]}}\ (\Cons{k}{t_l}), \rho \CPSTh{\Control{c}{e}} \LamP{v_0}{\Lam{t_0}{K[\Lam{t_1}{\Cons{k'}{t_1}}]}}\ t_r)\in R_{\beta}$と\\
(B)\ $(\rho \CPSTh{\Control{c}{e}}\ \LamP{v_0}{\Lam{t_0}{K[\Lam{t_1}{t_1}]}}\ t_l, \rho \CPSTh{\Control{c}{e}} \LamP{v_0}{\Lam{t_0}{K[\Lam{t_1}{t_1}]}}\ t_r)\in R_{\beta}$が成り立つ。\\
\\
(A)について、CPS変換と$\beta_{\Omega}$を使って以下のように変換できる。\\
(A)\ $(\DLet{x'}
      {\Lam{v}{\Lam{k'}{\Lam{t'}
        {\App{\App{\LamP{v_0}{\Lam{t_0}{K[\Lam{t_1}{t_1}]}}}{v}}{\DAppend{\ConsP{k}{t_l}}{\ConsP{k'}{t'}}}}}}}
      {\CPS{e}{\rho[\Change{c}{x'}]}{\Idk}{\Idt}},\\
\qquad   \DLet{x'}
      {\Lam{v}{\Lam{k'}{\Lam{t'}
        {\App{\App{\LamP{v_0}{\Lam{t_0}{K[\Lam{t_1}{\Cons{k'}{t_1}}]}}}{v}}{\DAppend{t_r}{\ConsP{k'}{t'}}}}}}}
      {\CPS{e}{\rho[\Change{c}{x'}]}{\Idk}{\Idt}})
      \in R_{\beta}$\\
\\
ここで、Agdaで示した補題\ref{ConsAssoc}より$\DAppend{\ConsP{k}{t_l}}{\ConsP{k'}{t'}} =_{\beta} \Cons{k}{(\DAppend{t_l}{\ConsP{k'}{t'}}})$である。\\
%% $(\DLet{x'}
%%       {\Lam{v}{\Lam{k'}{\Lam{t'}
%%         {\App{\App{\LamP{v_0}{\Lam{t_0}{K[\Lam{t_1}{t_1}]}}}{v}}{\Cons{k}{(\DAppend{t_l}{\ConsP{k'}{t'}}})}}}}}
%%       {\CPS{e}{\rho[\Change{c}{x'}]}{\Idk}{\Idt}},\\
%%    \DLet{x'}
%%       {\Lam{v}{\Lam{k'}{\Lam{t'}
%%         {\App{\App{\LamP{v_0}{\Lam{t_0}{K[\Lam{t_1}{\Cons{k'}{t_1}}]}}}{v}}{\DAppend{t_r}{\ConsP{k'}{t'}}}}}}}
%%       {\CPS{e}{\rho[\Change{c}{x'}]}{\Idk}{\Idt}})
%%       \in R_{\beta}$\\
      左右の$x'$の中身をそれぞれ$X,Y$とおくと、\\
      $(\CPS{e}{\rho[\Change{c}{X}]}{\Idk}{\Idt},\CPS{e}{\rho[\Change{c}{Y}]}{\Idk}{\Idt})\in R_{\beta}$を示せば良い。\\
\\
$e$に対しての帰納法の仮定より、\\
$x_i:\tau_i\ c:\tau \rightarrow t_1, (\mu_1) t_2 (\mu_2) \alpha \vdash e : \gamma (\mu_{id}) \gamma' (\Bullet) \beta$であり、かつ各$v_i$が$\vdash v_i:\tau_i$かつ$(v_i,v_i') \in R_{\tau_i}$で\\
$(k,k')\in k_{\gamma (\mu_{id}) \gamma'}$かつ$(\Lam{v_0}{\Lam{t_0}{K[]}}, \Lam{v_0}{\Lam{t_0}{K[]}})\in  K^{kk'}_{\gamma (\mu_{id}) \gamma'}$を満たすような任意のコンテキスト$K$と、$(t, t') \in T_{\Bullet}$を満たす任意の$t, t'$について、\\
$(\rho \CPSTh{e}\ \LamP{v_0}{\Lam{t_0}{K[\Lam{t_1}{t_1}]}}\ t, \rho \CPSTh{e} \LamP{v_0}{\Lam{t_0}{K[\Lam{t_1}{t_1}]}}\ t')\in R_{\beta}$が成り立つ。\\
\\
$\Lam{v_0}{\Lam{t_0}{K[f]}}\ =\ \Lam{v_0}{\Lam{t_0}{\Idk\ v_0\ (f\ t_0)}}$\\
上のコンテキストは$K^{kk'}_{\gamma (\mu_{id}) \gamma'}$を満たし(補題\ref{IdkContext})、$(t, t') \in T_{\Bullet}$より$t=t'=\Idt$なので
帰納法の仮定で(A)を導くには、$(X,Y)\in R_{\tau \rightarrow t_1, (\mu_1) t_2 (\mu_2) \alpha}$を示す。\\
%% $(\Lam{v}{\Lam{k'}{\Lam{t'}
%%     {\App{\App{\LamP{v_0}{\Lam{t_0}{K[\Lam{t_1}{t_1}]}}}{v}}{\Cons{k}{(\DAppend{t_l}{\ConsP{k'}{t'}}})}}}},\\
%%   \Lam{v}{\Lam{k'}{\Lam{t'}
%%       {\App{\App{\LamP{v_0}{\Lam{t_0}{K[\Lam{t_1}{\Cons{k'}{t_1}}]}}}{v}}{\DAppend{t_r}{\ConsP{k'}{t'}}}}}})\in R_{\tau \rightarrow t_1, (\mu_1) t_2 (\mu_2) \alpha}$\\
\\
$R_{\tau \rightarrow t_1, (\mu_1) t_2 (\mu_2) \alpha}$の定義から\\
$(V,V')\in R_{\tau}$と$(k_1,k_1')\in k_{t_1 (\mu_1) t_2}$かつ$(\Lam{v_0}{\Lam{t_0}{K[]}}, \Lam{v_0}{\Lam{t_0}{K[]}})\in K^{k_1k_1'}_{t_1 (\mu_1) t_2}$を満たすコンテキスト$K_1$と$(t_{l2},t_{r2})\in T_{\mu_1}$について\\
(C)\ $(X\ V\ \Lam{v_0}{\Lam{t_0}{K_1[\Lam{t_1}{t_1}]}}\ \ConsP{k_1}{t_{l2}},Y\ V'\ \Lam{v_0}{\Lam{t_0}{K_1[\Lam{t_1}{\Cons{k_1'}{t_1}}]}}\ t_{r2})\in R_{\alpha}$\\
(D)\ $(X\ V\ \Lam{v_0}{\Lam{t_0}{K_1[\Lam{t_1}{t_1}]}}\ t_{l2},Y\ V'\ \Lam{v_0}{\Lam{t_0}{K_1[\Lam{t_1}{t_1}]}}\ t_{r2})\in R_{\alpha}$を示せば良い。\\
\\
(C)について\\
X,Yを代入して補題\ref{Reduction2}より簡約できるので\\
$(\LamP{v_0}{\Lam{t_0}{K[\Lam{t_1}{t_1}]}}\ V\ \Cons{k}{(\DAppend{t_l}{\ConsP{\Lam{v_0}{\Lam{t_0}{K_1[\Lam{t_1}{t_1}]}}}{\ConsP{k_1}{t_{l2}}}}}),\\
\LamP{v_0}{\Lam{t_0}{K[\Lam{t_1}{\Cons{k'}{t_1}}]}}\ V'\ \DAppend{t_r}{\ConsP{\Lam{v_0}{\Lam{t_0}{K_1[\Lam{t_1}{\Cons{k_1'}{t_1}}]}}}{t_{r2}}})\in R_{\alpha}$を示す。\\
\\
$(\LamP{v_0}{\Lam{t_0}{K[]}},\LamP{v_0}{\Lam{t_0}{K[]}})\in K^{kk'}_{\tau(\mu_{\alpha})\alpha}$を満たしているから\\
$(V,V')\in R_{\tau}$と$(t, t')\in T_{\mu_{\alpha}}$について\\
$(\LamP{v_0}{\Lam{t_0}{K[\Lam{t_1}{t_1}]}}\ V\ \ConsP{k}{t},
\Lam{v_0}{\Lam{t_0}{K_1[\Lam{t_1}{\Cons{k'}{t_1}}]}}\ V'\ t')\in R_{\alpha}$\\
\\
そこで、\\
(E)\ $(\DAppend{t_l}{\ConsP{\Lam{v_0}{\Lam{t_0}{K_1[\Lam{t_1}{t_1}]}}}{\ConsP{k_1}{t_{l2}}}},
\DAppend{t_r}{\ConsP{\Lam{v_0}{\Lam{t_0}{K_1[\Lam{t_1}{\Cons{k_1'}{t_1}}]}}}{t_{r2}}})\in T_{\mu_{\alpha}}$を示す。\\
\\
ここで、$\Compatible{\Trail{t_1}{t_2}{\mu_1}}{\mu_2}{\mu_0}\quad \Compatible{\mu_\beta}{\mu_0}{\mu_\alpha}$である事を考慮して、\\
次のように$trail$を考える。\\
%%%%%%%%%%%%%%%%%%%%%%%%%%%%%%%%trail設定%%%%%%%%%%%%%%%%%%%%%%%%%%%%%%%%%%%%
$\mu_0 = \Trail{t_1}{t_2}{\mu_0}\quad \mu_2 = \Trail{\epsilon}{\epsilon'}{\mu'}$\\
$\mu_{\beta} = \Trail{\delta}{\delta'}{\mu}\quad \mu_{\alpha} = \Trail{\delta}{\delta'}{\mu_{\alpha}}$\\
%%%%%%%%%%%%%%%%%%%%%%%%%%%%%%%%trail設定%%%%%%%%%%%%%%%%%%%%%%%%%%%%%%%%%%%%
\\
(E)の$::$と$@$を展開する。
%% (E)\ $(\DAppend{t_l}{\Lam{v_1}{\Lam{t_1'}{\LamP{v_0}{\Lam{t_0}{K_1[\Lam{t_1}{t_1}]}}}\ v_1\ \Cons{\ConsP{k_1}{t_{l2}}}{t_1'}}},\\
%% \qquad \DAppend{t_r}{\Lam{v_1}{\Lam{t_1'}{\LamP{v_0}{\Lam{t_0}{K_1[\Lam{t_1}{\Cons{k_1'}{t_1}}]}}}\ v_1\ \ConsP{t_{r2}}{t_1'}}})\in T_{\mu_{\alpha}}$\\
%% 次に$@$を展開する。
また、$\mu_{\alpha} = \Trail{\delta}{\delta'}{\mu_{\alpha}}$なので\\
(E)\ $(\Lam{v_2}{\Lam{t_2'}{t_l\ v_2\ \Cons{\LamP{v_1}{\Lam{t_1'}{\LamP{v_0}{\Lam{t_0}{K_1[\Lam{t_1}{t_1}]}}}\ v_1\ \Cons{\ConsP{k_1}{t_{l2}}}{t_1'}}}{t_2'}}},\\
\qquad \Lam{v_2}{\Lam{t_2'}{t_r\ v_2\ \Cons{\LamP{v_1}{\Lam{t_1'}{\LamP{v_0}{\Lam{t_0}{K_1[\Lam{t_1}{\Cons{k_1'}{t_1}}]}}}\ v_1\ \ConsP{t_{r2}}{t_1'}}}{t_2'}}})\in T_{\delta(\mu_{\alpha})\delta'}$\\
\\
$T_{\delta(\mu_{\alpha})\delta'}$の定義から\\
$(V_{delta},V_{delta}')\in R_{\delta}$を満たす$V,V'$と$(t_3, t_4)\in T_{\mu_{\alpha}}$を満たす$t_3, t_4$について\\
$(t_l\ V_{delta}\ \Cons{\LamP{v_1}{\Lam{t_1'}{\LamP{v_0}{\Lam{t_0}{K_1[\Lam{t_1}{t_1}]}}}\ v_1\ \Cons{\ConsP{k_1}{t_{l2}}}{t_1'}}}{t_3},\\
t_r\ V_{delta}'\ \Cons{\LamP{v_1}{\Lam{t_1'}{\LamP{v_0}{\Lam{t_0}{K_1[\Lam{t_1}{\Cons{k_1'}{t_1}}]}}}\ v_1\ \ConsP{t_{r2}}{t_1'}}}{t_4})
\in R_{\delta'}$を示せば良い。\\
\\
ここで、$(t_l, t_r)\in T_{\mu_{\beta}}$つまり$(t_l, t_r)\in T_{\delta{\mu}\delta'}$なので定義より、\\
$(V_{delta},V_{delta}')\in R_{\delta}$を満たす$V,V'$とと$(t_{\mu}, t_{\mu}')\in T_{\mu}$を満たす$t_{\mu}, t_{\mu'}$について\\
$(t_l\ V_{delta}\ t_{\mu},t_r\ V_{delta}'\ t_{\mu}')\in R_{\delta'}$となるから\\
\\
(F)\ $(\Cons{\LamP{v_1}{\Lam{t_1'}{\LamP{v_0}{\Lam{t_0}{K_1[\Lam{t_1}{t_1}]}}}\ v_1\ \Cons{\ConsP{k_1}{t_{l2}}}{t_1'}}}{t_3},\\
\qquad \Cons{\LamP{v_1}{\Lam{t_1'}{\LamP{v_0}{\Lam{t_0}{K_1[\Lam{t_1}{\Cons{k_1'}{t_1}}]}}}\ v_1\ \ConsP{t_{r2}}{t_1'}}}{t_4})
\in T_{\mu}$を示せば良い。\\
\\
ここで、(F)の$::$を展開し\\
(F)\ $(\Lam{v_2}{\Lam{t_2'}{\LamP{v_1}{\Lam{t_1'}{\LamP{v_0}{\Lam{t_0}{K_1[\Lam{t_1}{t_1}]}}}\ v_1\ \Cons{\ConsP{k_1}{t_{l2}}}{t_1'}}}\ v_2\ \ConsP{t_3}{t_2'}},\\
\qquad \Lam{v_2}{\Lam{t_2'}{\LamP{v_1}{\Lam{t_1'}{\LamP{v_0}{\Lam{t_0}{K_1[\Lam{t_1}{\Cons{k_1'}{t_1}}]}}}\ v_1\ \ConsP{t_{r2}}{t_1'}}}\ v_2\ \ConsP{t_4}{t_2'}})\in T_{\mu}$となる。\\
\\
$\mu$について、\\
\begin{figure}[h]
\[
\begin{array}{lcl}
  \Compatible{\mu_\beta}{\mu_0}{\mu_\alpha} &=& \Compatible{\Trail{\delta}{\delta'}{\mu}}{\Trail{t_1}{\mu_0}{t_2}}{\Trail{\delta}{\mu_{\alpha}}{\delta'}}\\
  &=& \Compatible{\Trail{t_1}{\mu_0}{t_2}}{\mu_{\alpha}}{\mu}
\end{array}
\]
\caption{\textsf{comptible}展開1}
\label{Compatible1}
\end{figure}
図\ref{Compatible1}より$\mu=\Trail{t_1}{t_2}{\mu'}$と置くことができるので(F)は次のようになる。\\
\\
(F)\ $(\Lam{v_2}{\Lam{t_2'}{\LamP{v_1}{\Lam{t_1'}{\LamP{v_0}{\Lam{t_0}{K_1[\Lam{t_1}{t_1}]}}}\ v_1\ \Cons{\ConsP{k_1}{t_{l2}}}{t_1'}}}\ v_2\ \ConsP{t_3}{t_2'}},\\
\qquad \Lam{v_2}{\Lam{t_2'}{\LamP{v_1}{\Lam{t_1'}{\LamP{v_0}{\Lam{t_0}{K_1[\Lam{t_1}{\Cons{k_1'}{t_1}}]}}}\ v_1\ \ConsP{t_{r2}}{t_1'}}}\ v_2\ \ConsP{t_4}{t_2'}})\in T_{t_1(\mu')t_2}$を示す。\\
定義より、\\
$(V_1,V_1')\in R_{t_1}$と$(t_{\mu_2},t_{\mu_2}')\in T_{\mu'}$について\\
(G)\ $(\LamP{v_0}{\Lam{t_0}{K_1[\Lam{t_1}{t_1}]}}\ V_1\ \Cons{\ConsP{k_1}{t_{l2}}}{\ConsP{t_3}{t_{\mu_2}}},\\
\qquad \LamP{v_0}{\Lam{t_0}{K_1[\Lam{t_1}{\Cons{k_1'}{t_1}}]}}\ V_1'\ \Cons{t_{r2}}{\ConsP{t_4}{t_{\mu_2}'}})\in R_{t_2}$を示せば良い。\\
\\
ここで、補題\ref{ConsAssoc}より、$\Cons{\ConsP{k_1}{t_{l2}}}{\ConsP{t_3}{t_{\mu_2}}} =_{\beta} \Cons{k_1}{\ConsP{t_{l2}}{\ConsP{t_3}{t_{\mu_2}}}}$なので\\
(G)\ $(\LamP{v_0}{\Lam{t_0}{K_1[\Lam{t_1}{t_1}]}}\ V_1\ \Cons{k_1}{\ConsP{t_{l2}}{\ConsP{t_3}{t_{\mu_2}}}},\\
\qquad \LamP{v_0}{\Lam{t_0}{K_1[\Lam{t_1}{\Cons{k_1'}{t_1}}]}}\ V_1'\ \Cons{t_{r2}}{\ConsP{t_4}{t_{\mu_2}'}})\in R_{t_2}$を示せば良い。\\
\\
ここで、$K_1$が$(\Lam{v_0}{\Lam{t_0}{K[]}}, \Lam{v_0}{\Lam{t_0}{K[]}})\in K^{k_1k_1'}_{t_1 (\mu_1) t_2}$を満たすので、\\
$(V_1,V_1')\in R_{t_1}$と$(t,t')\in T_{\mu_1}$について\\
$(\LamP{v_0}{\Lam{t_0}{K_1[\Lam{t_1}{t_1}]}}\ V_1\ \ConsP{k_1}{t}, \LamP{v_0}{\Lam{t_0}{K_1[\Lam{t_1}{\Cons{k_1'}{t_1}}]}}\ V_1'\ t)
\in R_{\tau_2}$が言える。\\
\\
つまり、(H)\ $(\ConsP{t_{l2}}{\ConsP{t_3}{t_{\mu_2}}},\Cons{t_{r2}}{\ConsP{t_4}{t_{\mu_2}'}})\in T_{\mu_1}$を示せば良い。
\\
ここで、\\
\begin{figure}[h]
\[
\begin{array}{lcl}
  \Compatible{\Trail{t_1}{t_2}{\mu_1}}{\mu_2}{\mu_0} &=& \Compatible{\Trail{t_1}{t_2}{\mu_1}}{\mu_2}{\Trail{t_1}{t_2}{\mu_0}}\\
  &=& \Compatible{\mu_2}{\mu_0}{\mu_1}
\end{array}
\]
\caption{\textsf{comptible}展開2}
\label{Compatible2}
\end{figure}\\
\\
また、図\ref{Compatible1}の続きから\\
\begin{figure}[h]
\[
\begin{array}{lcl}
  \Compatible{\Trail{t_1}{\mu_0}{t_2}}{\mu_{\alpha}}{\mu} &=& \Compatible{\Trail{t_1}{\mu_0}{t_2}}{\mu_{\alpha}}{\Trail{t_1}{t_2}{\mu'}}\\
  &=& \Compatible{\mu_{\alpha}}{\mu'}{\mu_0}
\end{array}
\]
\caption{\textsf{comptible}展開3}
\label{Compatible3}
\end{figure}\\
\\
(H)を示す。\\
$(t_3,t_4)\in T_{\mu_{\alpha}}\ (t_{\mu_2},t_{\mu_2}')\in T_{\mu'}$で図\ref{Compatible3}より$\Compatible{\mu_{\alpha}}{\mu'}{\mu_0}$\\
補題\ref{TCompatible}より$(\Cons{t_3}{t_{\mu_2}},\Cons{t_4}{t_{\mu_2}'})\in T_{\mu_0}$\\
また、$(t_{l2},t_{r2})\in T_{\mu_2}$で図\ref{Compatible2}より$\Compatible{\mu_2}{\mu_0}{\mu_1}$なので\\
補題\ref{TCompatible}より$(\Cons{t_{l2}}{\ConsP{t_3}{t_{\mu_2}}},\Cons{t_{r2}}{\ConsP{t_4}{t_{\mu_2}'}})\in T_{\mu_1}$となる。\\
これで(H)が示せた。\\
(B),(D)についても同様に示す。\\


\subsection{補題\ref{KMove}の証明}
定理\ref{HandMain}を使って補題\ref{KMove}の\\
$(\DLam{v}{\DLam{k}{\DLam{t}{\CPSTh{e}[\Change{x}{v}] \SLamP{v_0}{\SLam{t_0}{\DApp{\Idk}{\DApp{v_0}{t_0}}}}\ \ConsP{k}{t}}}},\\
\ \DLam{v}{\DLam{k}{\DLam{t}{\CPSTh{e}[\Change{x}{v}] \SLamP{v_0}{\SLam{t_0}{\SApp{\Idk}{\SApp{v_0}{\ConsP{k}{t_0}}}}}\ t}}})
\in R_{\tau_2 \rightarrow \TrailType{\alpha} \TrailType{\beta}}$を示していく。\\
\\
$(\Lam{v'}{\Lam{k'}{\Lam{t'}{\CPSTh{e}[\Change{x}{v'}] \LamP{v_0}{\Lam{t_0}{K[]}}\ \ConsP{k'}{t'}}}},\\
\Lam{v'}{\Lam{k'}{\Lam{t'}{\CPSTh{e}[\Change{x}{v'}] \LamP{v_0}{\Lam{t_0}{K[]}}\ t'}}})
\in R_{\tau_2 \rightarrow \TrailType{\alpha} \TrailType{\beta}}$をまず考える。
\\
$(k, k)\in T_{\mu_k}\ (V,V')\in R_{\tau_2}\  (\LamP{v_0}{\Lam{t_0}{K[]}},\LamP{v_0}{\Lam{t_0}{K[]}})\in K^k_{\tau_1(\mu_{\alpha})\alpha}\  (t_2,t_2')\in T_{\mu_{\beta}}$について、\\
\\
(A)\ $(\LamP{v'}{\Lam{k'}{\Lam{t'}{\CPSTh{e}[\Change{x}{v'}] \LamP{v_0}{\Lam{t_0}{K[\Lam{t_1}{t_1}]}}\ \ConsP{k'}{t'}}}}\ V\ k\ t_2,\\
\qquad \Lam{v'}{\Lam{k'}{\Lam{t'}{\CPSTh{e}[\Change{x}{v'}] \LamP{v_0}{\Lam{t_0}{K[\Lam{t_1}{\Cons{k'}{t_1}}]}}\ t'}}}\ V'\ k\ t_2')\in R_{\beta}$と\\
(B)\ $(\LamP{v'}{\Lam{k'}{\Lam{t'}{\CPSTh{e}[\Change{x}{v'}] \LamP{v_0}{\Lam{t_0}{K[\Lam{t_1}{t_1}]}}\ \ConsP{k'}{t'}}}}\ V\ k\ t_2,\\
\qquad \Lam{v'}{\Lam{k'}{\Lam{t'}{\CPSTh{e}[\Change{x}{v'}] \LamP{v_0}{\Lam{t_0}{K[\Lam{t_1}{t_1}]}}\ t'}}}\ V'\ k\ t_2')\in R_{\beta}$を示せば良い。\\
補題\ref{Reduction2}より簡約できるので\\
(A)\ $(\rho \CPSTh{e}[\Change{x}{V}] \LamP{v_0}{\Lam{t_0}{K[\Lam{t_1}{t_1}]}}\ \ConsP{k}{t_2},
 \rho \CPSTh{e}[\Change{x}{V'}] \LamP{v_0}{\Lam{t_0}{K[\Lam{t_1}{\Cons{k}{t_1}}]}}\ t_2')\in R_{\beta}$\\
(B)\ $(\rho \CPSTh{e}[\Change{x}{V}] \LamP{v_0}{\Lam{t_0}{K[\Lam{t_1}{t_1}]}}\ t_2,
 \rho \CPSTh{e}[\Change{x}{V'}] \LamP{v_0}{\Lam{t_0}{K[\Lam{t_1}{t_1}]}}\ t_2')\in R_{\beta}$\\
\\
定理\ref{HandMain}より、\\
(A)'\ $(\rho \CPSTh{e}[\Change{x}{V}] \LamP{v_0}{\Lam{t_0}{K[\Lam{t_1}{t_1}]}}\ \ConsP{k}{t_2}, \rho \CPSTh{e}[\Change{x}{V'}] \LamP{v_0}{\Lam{t_0}{K[\Lam{t_1}{\Cons{k}{t_1}}]}}\ t_2')\in R_{\beta}$と\\
(B)'\  $(\rho \CPSTh{e}[\Change{x}{V}] \LamP{v_0}{\Lam{t_0}{K[\Lam{t_1}{t_1}]}}\ t_2, \rho \CPSTh{e}[\Change{x}{V'}] \LamP{v_0}{\Lam{t_0}{K[\Lam{t_1}{t_1}]}}\ t_2')\in R_{\beta}$が成り立つ。\\
これで\\
$(\Lam{v'}{\Lam{k'}{\Lam{t'}{\CPSTh{e}[\Change{x}{v'}] \LamP{v_0}{\Lam{t_0}{K[]}}\ \ConsP{k'}{t'}}}},\\
\Lam{v'}{\Lam{k'}{\Lam{t'}{\CPSTh{e}[\Change{x}{v'}] \LamP{v_0}{\Lam{t_0}{K[]}}\ t'}}})
\in R_{\tau_2 \rightarrow \TrailType{\alpha} \TrailType{\beta}}$が示せた。
\\
補題\ref{IdkContext}により以下のようになり、補題\ref{KMove}が示せた。\\
$(\Lam{v'}{\Lam{k'}{\Lam{t'}{\CPSTh{e}[\Change{x}{v'}] \LamP{v_0}{\LamP{t_0}{\SApp{\Idk}{\SApp{v_0}{t_0}}}}\ \ConsP{k'}{t'}}}},\\
\Lam{v'}{\Lam{k'}{\Lam{t'}{\CPSTh{e}[\Change{x}{v'}] \LamP{v_0}{\LamP{t_0}{\SApp{\Idk}{\SApp{v_0}{\ConsP{k'}{t_0}}}}}\ t'}}})
\in R_{\tau_2 \rightarrow \TrailType{\alpha} \TrailType{\beta}}$\\


