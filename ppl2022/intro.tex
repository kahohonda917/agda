\section{はじめに}
継続とは、ある時点における残りの計算のことを意味する。例えば、1 + (2 * 3)という計算で(2 * 3)を実行しているときの継続は、「2 * 3の結果を受け取って、そこに1を加える」ということになる。プログラムの中で継続を扱う方法として、プログラム全体を継続渡し形式(continuation-passing style; CPS)に変換するというものがある。

本稿は、単純型付きλ計算に限定継続演算子であるcontrol/prompt\cite{POPL88}を加えた体系でのCPS変換の証明の定式化を目指している。control/promptは四種ある限定継続演算子のうちの一つで、controlが「その時点での継続を切り取ってくる命令」、promptが「切り取られる継続の範囲を限定する命令」である。同じく限定継続演算子の一種であるshift/reset\cite{DF1990,DF1992} においてはCPS変換の証明が定式化されており、本稿ではそれをcontrol/promptに拡張する。

control/promptはshift/resetの定式化のときに加えて、継続が連なったtrailという概念が加わっているため更に持っている型が増えている。例えば、shift/resetの際は$\Arrow{\tau}{\tau}$と二つの型で表現されていた継続は、control/promptではtrailが入ったことにより$\Arrow{\tau_1}{\Arrow{\mu_{\alpha}}{\alpha}}$という三つの型で表現される。
trailの構造に対しても場合分けが必要になり、かなり複雑になるため、
間違いを防ぐために Agda 上で証明を行なった。
変数束縛には PHOAS \cite{chlipala-phoas} を用いることで、
$\alpha$ 変換などの定式化をすることなく、比較的、扱いやすい形で
定式化を行った。
ただし、証明の途中で特定の形の項が同値であることを示す必要が出てきたた
め、そこで論理関係を用いた。
この部分は PHOAS とは相性が悪く Agda にのせることが困難であったため、
手で証明を行なった。
その模様を報告する。

以下、2節でcontrol/prompt入り単純型付き$\lambda$計算の定式化を、3節でCPS変換を定式化した上で、4節でCPS変換の正当性を証明する。また、5節で関連研究を、6節でまとめと今後の課題を述べる。
