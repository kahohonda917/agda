\section{関連研究}
control/prompt に対する型システムと CPS 変換の正当性の証明は Kameyama,
Yonezawa \cite{KY2008} が与えている。しかし、彼らの CPS 変換は trail
の実装にリストを使っているのに対して、本論文では高階の関数を使用してい
る。また、彼らの型システムでは全ての trail が同じ型を持たなくてはなら
ないのに対し、本論文の型システムにはそのような制約はない。(両者の詳細
な比較は \cite{FSCD2021} を参照されたい。)そのため、正当性の証明の大
枠は両者で似たものになっているが、trail の型の整合性の取り方など特有の
手法が必要となっている。

これまで我々のグループでは let 多相の入ったラムダ計算に対する CPS 変換
の正当性の証明 \cite{CHAM2018} や、shift/reset の入った体系に対する選
択的な CPS 変換の証明 \cite{CHAM2020} などを Agda で型付きで定式化して
きた。これらに比べ、本論文での証明は論理関係を使った証明を必要としてい
たため、一部を Agda にのせることができていない。それを Agda にのせるの
は今後の課題だが、環境を明示的に扱うのは PHOAS \cite{chlipala-phoas}
による項の表現を使っていると難しいと予想されるため、簡単ではないと思わ
れる。de Bruijn index を使うなどの方法は考えられるが、証明はとても大き
くなることが予想され \cite{YK2003}、control/prompt の体系のように証明
が大きくなると現実的ではない可能性がある。

%Isabelle/HOL による$\alpha$同値性を使った定式化\cite{YK2003} 、Coq による de Bruijn index を使った定式化\cite{ZX2007}、Twelf による HOAS を用いた定式化\cite{YH2006} がある。

%限定継続演算子の型システムにおいては、CPSを基に構築する手法が採られてきた。その中にshift/reset\cite{DF1989,AK2007},control/prompt\cite{KY2008}がある。Danvy, Filinski\cite{DF1989}は全ての限定継続演算子に対応できる型システムを考えている。

%CPS変換はこれまで様々な形で実装されてきたが、Isabelle/HOL による$\alpha$同値性を使った定式化\cite{YK2003} 、Coq による de Bruijn index を使った定式化\cite{ZX2007}、Twelf による HOAS を用いた定式化\cite{YH2006} がある。

%CPS変換の正当性をshift/reset入りの体系で示した論文に\cite{CHAM2020}が挙げられる。また、\cite{CHAM2018}は単純型付き$\lambda$計算に\textsf{let}多相が入った体系のための正当性の証明である。
